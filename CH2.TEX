%Documentation [2nd draft]
\chapter{Inverting Real Lidar Returns: Practical Considerations and
Caveats}

\section{Data Quality}

Figure~{2.1} \footnote{Many of the following figures in this section 
have been produced from data processed
by the the software package KNMI2B2 developed for this project 
and to be described in the next chapter. 

The package uses the algorithms and techniques previously 
described for and analyzing cloud lidar data. The software 
package Origin by MicroCal Software Inc. a Windows 3.0
application has been used to present the data here. 
Please consult the documentation that accompanies
Origin and MicroCal for more details.} shows the expected lidar 
signal return at 
1064$nm$ calculated by assuming a pure Rayliegh 
atmosphere as an attentuating and backscattering
medium, calibrated and plotted with the data from a cloudless sky
on August 4th, 1993 from the RIVM lidar system. This calibration
in effect serves to provide our theoretical lidar signal with
the constants at the front of the lidar equation, Eq.~{\ref{lideq}}. 

\begin{figure}
\vspace{5.0in}
\caption{expected lidar signal return at 
1064$nm$ calculated by assuming a pure Rayliegh 
atmosphere as a attentuating and backscattering
medium, calibrated and plotted with the data from a cloudless sky
on August 4th, 1993 from the RIVM lidar system}
\end{figure}

We note that the actual return is less than optimal beyond
a range of approximately 1200$m$. This shortfall can be attributed
to a number of factors not taken into consideration in the
calculation of the signal.

By far the most important factor that has not been included
in this projection is the presence of atmospheric aerosols 
in the boundary layer. From the authors experience, these typically 
have optical extinction values on the
order of ${10^{-2}}{1\over{km}}$. This strong attenuating 
presence is a possible explaination for much of the losses 
early in the profile leading to this rather modest range depth for the system. 

% Although the RIVM has been collecting such data for over
% a decade, specifically to characterize the boundary layer height 
% and to obtain aerosol profiles, very little analyses have been
% performed. Such analyses would allow us to determine a typical
% aerosol distribution, which we could use to establish some
% estimation of expected optical-radiative contribution and
% account for it in our system performance prediction. These analyses
% were not performed by us as it was not entailed by the defined scope
% of our work. In fact, evaluations of system performance are also
% the domain of the system designers and operators.
 
As the system is an {\em aerosol boundary layer} lidar, the optics,
transmitter and receiver have not been {\em optimized}
for clouds. The amount of adjustment required to recently realign 
a similar RIVM system for clouds can atest to the marked difference
between the two configurations \cite{evpd}. Furthermore, any digitizing
system has inherent limits in its ability to resolve low signal levels,
and this is clearly the case for regions beyond 1200$m$.

Complicating the issue further is the presence of {\em after-pulsing}
or {\em ringing} in the detectors when exposed to high signals.
Figure~{2.2} shows two individual backscatter profiles (74 and 225)
taken from the same day, August 4th, 1993 [Julian date:93216]. 

\begin{figure}
\vspace{5.0in}
\caption{Lidar backscatter profiles from August 4th, 1993 [93216], 
trace 74 {\em(top)} and trace 225 {\em(bottom)}.}
\end{figure}

Profile 74 clearly
represents a saturated cloud return, with a signal value maximum of about
32000. Note that although this maximum value is located far beyond the height
we expect to detect any signal from the background atmosphere there
appears to be some signal above the cloud which is
higher than the signal of the background atmosphere prior to entry
into the cloud. 

Compare this to profile 225, with a much lower cloud peak signal 
(6000) which is clearly not
saturated and is located below 1500$m$, within what we expect to 
be the range at which the background atmosphere will be detectable
by the system. The signal above it clearly is lower than the
signal prior to entry into the cloud. This is the result of
attenuation of the incident and return pulse by the intervening
medium, specifically the cloud, as we expect. 

While it is possible that there is a strongly backscattering
aerosol or haze layer above the cloud in trace 74, more than likely
what is being observed are the effects of after pulsing or ringing
from the photodetectors in question, avalanche photodiodes (APD).
This is quite a common problem with such detectors\cite{jacek}. 
It was initially expected at the start of this study, 
by the technical personnel primarily involved with the system, that
the recovery time of the detectors was smaller than the
time resolution of the data acquisition components. Recently
it has however been confirmed that this is not the case and that
we are observing after-pulsing in our recorded data\cite{evpd}. 

The accuracy of the Klett inversion is dependent upon the acuuracy
of the boundary condition, the reference optical extinction. As was
previously discussed, unless it is otherwise known, the application
of the inversion requires an estimate. The reference height is selected
above the cloud, with the expectation that the only optically scattering or
attenuating constituent at that height will be atmospheric molecules. 
If a sonde measurement is taken providing the pressure and temperature
at or near that height, the optical extinction can be calculated and
used as our boundary condition.

However, if after-pulsing occurs, the signal values above the cloud
are then not the result of backscattering from a molecular atmosphere,
but an artifact created by the detection system.
Hence, it is advised that a reference height be chosen sufficiently
high enough above the cloud that one is confident that the inversion
starts with values that are backscatter and not after-pulsing. 
Furthermore, near saturated data should be approached with caution
and of course, saturated data discarded.
 
In Table~{2.1} are some techincal details relevant to our work regarding
the lidar system under question.

\begin{table}
\caption{Technical Overview of the
RIVM Truck Mounted Nd:YAG 1064 nm,
70 mJ Aerosol Boundary Layer Lidar System}
\begin{center}
\begin{tabular}{|l|l|}
\hline
\multicolumn{2}{|c|}{Transmitter}\\
\hline
Wavelength & 1064 $nm$ \\
\hline
Pulse repetition & 10 $Hz$ \\
\hline
Pulse length & 9 $ns$ \\
\hline
Pulse energy & 70 $mJ$ at 1064 $nm$ \\
(simultaneous output) &  \\
\hline
Beam divergence & 1 $mrad$ full angle \\
\hline
\multicolumn{2}{|c|}{Receiver}\\
\hline
Telescope & 32 cm (diameter) Fresnel lens \\
field of view (FOV) & 3.1 $mrad$ full angle \\
\hline
System FOV  & 0.5 $mrad$\\
\hline 
Optics & 2\% transmission neutral density \\
\hline
Detectors & APD at 1064 nm \\
\hline
Signal processing & transient recorder \\
\hline
\multicolumn{2}{|c|}{Data acquisition}\\ 
\hline
\multicolumn{2}{|c|}{transient digitizer sample $f$= 100 $MHz$}\\ 
\multicolumn{2}{|c|}{averaging: 250 shots/25 sec. 
every 4 minutes; 24 hours/day}\\ 
\multicolumn{2}{|c|}{other: background subtraction using pretrigger}\\
\multicolumn{2}{|c|}{pretrigger length: 5$ms$ (500 analog data points)}\\ 
\multicolumn{2}{|c|}{data: range (index) corrected, scaling factor (uknown)}\\
\hline
\end{tabular}
\end{center}
\end{table}

\section{Practical Application of the Klett Inversion Algorithm}

In our work only a basic rectangular numerical integration was performed. 
As we are dealing with relatively large data sets, previous studies 
have shown that a more sophisticated routine 
does little to improve our results \cite{agc}.
Furthermore, since we have derived this solution from
Eq.~{\ref{lideq}}, as we have discussed in Sec.~{2.1.5}
such an inversion will not account for
multiple scattering effects. Monte-Carlo 
simulations indicate this deficiency can only account for a 
contribution on the order of 10 percent. In fact, for clouds of
low optical depths, such as most cirrus, 
multiple scattering contributions can be negligible \cite{cmrpacd2} 
In general $C$ and $k$ in Eq.~{\ref{betasig2}} can be range dependent however
it has been found that good inversions can be obtained by using constant
values for these quantities. We discuss the potential for taking into
account a variable backscatter to extinction ratio instead of
$C$ in the final section of this report.

In general, $k$ has been set to unity,
and it has been accepted that there are no {\em stringent} reasons
for any other choice. It has also been reported that extinction
inversions such as {\em Klett's}, depend only weakly on $k$.\cite{wcrr}
In most cases, both the constant of proportionality and
$k$ are taken as constant along the path of inversion.
Mulders \cite{jmm}, has found them to 
vary greatly in the same temporal and spatial fields.
Studies of the sensitivity of Klett's inversion to the parameters
have shown small variations in $k$ to have little effect,
except in cases where $\sigma(r)$ is at or near singularity \cite{hghjafdhs}.
Varying $k$ to suit the expected cloud can be difficult
and sometimes impossible to determine for real atmospheric conditions.
%
% Cirrus are predominantly composed of a
% variety of ice crystal habits with complex structures such as bullet shaped 
% cylinders, rosettes and hexagonal columns and the like, which are thought to
% behave like isotropic scatterers such that the value of $k$ will be relatively
% independent of size and shape \cite{rrr}.
Past efforts, using varying $k$ values have shown very minor improvements
while involving a major increase in computational time. \cite{agc}
% taking multiple scattering considerations
% into account for $k$ and the constant of proportionality
% have shown a required increase in the processing time for inversions, yielding
% comparatively minor improvements in the accuracy of our results.
% 
\subsection{Estimations of the Reference Altitude and Extinction}

\begin{figure}
\vspace{5.0in}
\caption{Graph showing optical extinction
profiles for a pure molecular or Rayleigh atmosphere at
$1064nm$. Generated using sonde measurements \SOLID and using 
an assumed lapse rate of ${-6.5^{o}}{C/km}$ and the
hydrostatic pressure equation \DASHED}
\end{figure}

For the far-end inversions, the reference 
altitude for $\sigma_{f}$ is 
% as conventionally 
chosen as high as possible, as long as there is a good SNR.
% without venturing too far into ranges of completely degraded signal and overwhelming
% orders of bit-noise. 
In doing so, errors incurred by inaccuracies in the $\sigma_{f}$ value 
% in our extinction coefficient estimate are allowed to 
decay to a reasonable level before approaching the lower 
altitude region of primary interest.
% range values where our lidar return signal is
% strong, and therefore of interest to us. The physical and mathematical basis
% for such an assumption can be shown on the basis of the requisite transmission
% functions \cite{agc}. 

% In our case, an altitude in the range of $12-15\;km$ was selected,
% since in most cases the clouds are below 
% this altitude, and we could establish the clear atmosphere 
% $\sigma_{f}$ values quite well.

The reference height is chosen high enough to be relatively safe in assuming
that backscatter from this altitude  was predominantly from a molecular
or a Rayleigh atmosphere. Figure~{2.3} is a plot showing a optical extinction
profiles for a pure molecular or Rayleigh atmosphere at the $1064nm$. 
wavelength. We have generated
one profile using an assumed lapse rate of ${-6.75^{o}}{C/km}$ and the
hydrostatic pressure equation.

It should be noted that backscatter contributions from molecular as well as
aerosol atmospheric constituents are relatively low at the $1064nm$
wavelength, hence we would expect very low SNR as such. This results
in difficulties in specifying reference values for Klettt Inversion as
we will discuss in length. This low backscatter is the result of a
rather small cross-section at in the infra-red region. At $1064nm$,
this translates into an area of ${10^{-32}}{m^2}$, apporximately
one order of magnitude less than in the visible at $532nm$. 
One can conclude immediately, that such a consideration would
indicate that a visible wavelength would be preferable for
optical measurements using lidar, and as well will show, this
is largely the case.

% The accuracy of this inversion algorithm is largely 
% determined by the accuracy of this boundary value estimate.
Based on the assumption that aerosol and molecular components are the 
only constituents at a chosen reference altitude a reference extinction
value can be calculated using a molecular atmosphere.
The Rayleigh calculations can utilize meteorological parameters 
obtained from sonde measurements.

% with some assurance from the Rayleigh and Mie expression for 
% including Shettle and Fenn's aerosol
% model for the appropriate season \cite{epsrwf}
% whose validity
% we discuss later. Despite it's widespread use, the Klett inversion's
% accuracy in practical applications is limited because of the heavy 
% dependence on boundary conditions which are, by necessity, 
% estimates \cite{aamruwcwwm}.

The success and failure of the Klett Inversion depends on a number of
factors, however Klett \cite{jdk1} identifies the most 
critical as being the selection of
an accurate as possible reference value for the extinction {\em (boundary
value)}. 

From the work performed on the RIVM data set,
we have found that proper selection of a reference signal value and
reference height are also important considerations.

\begin{figure}
\vspace{5.0in}
\caption{Simulated relative range-corrected logarithmic 
lidar backscatter signal $(S-S_{0})$
for a simulated {\em plateau shaped} optical extinction profile}
\end{figure}

\begin{figure}
\vspace{5.0in}
\caption{Klett inversions of the previous simulated 
backscatter profile figure using varying values
for the reference optical extinction boundary condition,
$\sigma_{f}$.}
\end{figure}

Figure~{2.4-2.5} shows a simulated relative range-corrected logarithmic 
lidar backscatter signal for a 
simulated {\em plateau shaped} optical extinction profile and
an Klett inversions of the backscatter using varying values
for the reference optical extinction boundary condition,
respectively. This is similar to the simulation performed
in Kletts original article describing this techinique\cite{jdk1}.

From Figure~{2.5}, it is apparent that the inversion is quite
{\em forgiving}, with respect to ucertainty in the estimation
of the reference optical extinction boundary condtion, 
$\sigma_{f}$. The correct value $\sigma_{f}$ is $1\over{km}$,
however, estimations as high as 100\% above and 50\% below
the correct value still result in a convergence to the correct
profile.

\begin{figure}
\vspace{5.0in}
\caption{Klett inversion of a simulated
constant extinction profile using varying values
of $\sigma_{f}$.}
\end{figure}

Figure~{2.6} is a similar result but for the case of a simulated
constant extinction profile, from which it is easier to gauge
the effects of errors in our boundary condition estimate of
$\sigma_{f}$. It can be seen that the inversion has corrected itself to within
10\% of the actual value regardless of high above or below we
have estimated $\sigma_{f}$ from the correct value. However
it should be noted that errors in the inversion which result from
an underestimation of $\sigma_{f}$ tend to propogate further
down the length of the profile and are hence more detrimental. 

\begin{figure}
\vspace{5.0in}
\caption{Klett inversion of a simulated
constant extinction profile using varying values
of reference height signals.}
\end{figure}

\begin{figure}
\vspace{5.0in}
\caption{Klett inversion of a simulated
constant extinction profile with varying
constant values added and subtracted to the entire profile.}
\end{figure}

Similar analyses have been performed for errors in the reference
height signal value and addition and subtraction of constant values
to the entire profile and are shown in Figures~{2.7-2.8}, respectively.

As previously, discussed, we choose our reference 
height at which we start our inversion as high as possible.
Since the detected atmospheric backscatter at high altitudes from 
the system is low or nonexistent, we expect that these reference 
signal values will most likely be erroneus data which are nothing
more than pure noise from the detector electronics. Thus an
assessment of the effects of an incorrect value in the reference
signal must be obtained. From Fig.~{2.7}, we see values below
the actual signal result in initial over prediction in the optical
extinction, but which correct themselves rather quickly.
For values above the correct signal we see an initial underestimate
in the optical extinction which is slow to recover to the correct 
profile values.

The latter analysis, Fig.~{2.8}, 
was performed with an interest in assessing the
effects of {\em background subtraction}. 
This is a common practice
in the processing of experimental lidar data. A value for noise 
which results from the system data acquisition electronics etc.,
is obtained from a portion of the digitized signal either before
or after the active triggering range in which the atmospheric 
measurement takes place. This value ussually an average of the
signal over this pre or post trigger range and is subtracted
from the entire profile, so as to obtain a signal profile which
is {\bf\em ideally} the result of only the detected atmospheric backscatter. 

For noisy signals this often results in zero-crossings of the
backscatter signal values. This is not permitted by our inversion 
mathematically as we must define the logarithmic-range adjusted
power, Eq.~{\ref{SR}}, which will then be undefined for such values. 
In our case, we know that our signals are quite noisy, specifically
beyond the 1200$m$ range, and expect many zero crossings. In such
a non-ideal case it is sometimes useful to reconstitute the 
original backscatter signal by re-adding this background noise value
to the profile. If this value is not known or if the data 
acquisition component records spurious negative values for low SNR
regions of the data then it is useful to add some constant
to the profile also.

\begin{figure}
\vspace{5.0in}
\caption{Graph of range at which varying 
the reference optical extinction value, reference height signal value 
and constant added or subtracted to the profile effect of the height 
at which the recovery of the inversion to within 10\% of the correct
value occurs.}
\end{figure}

From Fig.~{2.8}, we observe that subtracting a constant value from 
the actual signal results in initial over prediction in the optical
extinction, but which correct themselves rather quickly.
Adding constant values to the correct signal however, results 
in an initial underestimate in the optical extinction which 
is slow to recover to the correct profile values.

To gauge the relative contributions of varying all these boundary
conditions, Fig.~{2.9} shows the range at which the recovery of the 
inversion to within 10\% of the correct value occurs when varying 
the reference optical extinction value, reference height signal value 
and a constant is added or subtracted to the profile.

It is seen that the inversion is relatively robust to 
positive uncertainties in the reference extinction value chosen
whereas underestimation of this value tends to have more far reaching 
effects as we previously mentioned. The algorithm is far more
sensitive to errors in reference height signal and to background addition.
An underestimation of reference height signal tends to recover
faster than an overestimation and background addition or
subtraction tends not to affect the inversion provided
the constants added or subtracted are sufficiently small
(within 100\% of the signal value).

Upon rexamination of the inversion algorithm, 
Eq.~{\ref{kfee}}, the reason behind 
such attributes of the inversion becomes clear. For large
positive values ($\sigma_{f}\rightarrow\infty$) 
of the boundary condition term, 
$1\over{\sigma_{f}}$, in the denominator
of Eq.~{\ref{kfee}}, the contribution to the algorithm
is quite small. As $\sigma_f$ becomes small however ($\sigma_{f}\rightarrow 0$),
this term bcomes much larger in the denominator leading
to underestimations of the extinction.

It can also be seen from Eq.~{\ref{kfee}} that the 
reference height signal given as 
a logarithmic range adjusted power, $S_{f}$, affects
the inversion throughout. 

The recoverablity of the inversion is largely due to the 
introduction of higher and more accurate values further
downstream in the signal which dominate the inversion 
and bring it towards
the correct values through terms such as the integral in the
denominator. Although such uncertainties become less and
less significant as the inversion proceeds, it is important
to note that, again, because of the integral term, noise
and other errors in the data have some, if not modest 
effect throughout. It is there always in our best interest 
to reduce such effects through averaging or any other method available.
It should be noted that the propogative effects of noise are not
discussed here but have been covered elsewhere \cite{jdk1}.

In general, it is the relationship between the reference height signal, 
$S_{f}$, and the reference extinction coefficient, $\sigma_{f}$,
that {\em calibrates} the inversion of the signal throughout.
A grossly inaccurate estimate of both can lead to potentially disasterously
inaccurate results, especially for values near the reference height.

\subsection{Use of Averaging Techniques}

\begin{figure}
\vspace{5.0in}
\caption{Plot of an original lidar backscatter signal as collected 
from August 4 1993 [93216] ({\DOTTED}). A profile where the value of the
most negative signal value is added to the entire profile along with
a small offset is also plotted ({\DASHED}). Signal values from
the topmost range height $(4000 m)$ down to $1500 m$ (the height 
at which we expect to begin to detect some of the background 
atmosphere) are averaged or fitted to 5 points above and below
the signal point of interest and shown as ({\SOLID}).}
\end{figure}

The conventional averaging method to improve SNR is with respect
to time, as we have previously mentioned. 
This is not adviseable for our data since both clouds are
extremely variable in time and our duty cycle of four minutes
between data acquisition cycles is too long. Clouds that are
a seen in one signal trace can completely dissappear within
the next subsequent profile. Hence averaging such consecutive
returns would result in a meaningless profile.

Should our data acquisition protocol change to a more consecutive
and continuous procedure this may be possible. This method
effective extends the exposure time of the detectors without
degrading resolution and causing saturation. There have been
some encouraging indications from a similar system 
at the RIVM recently modified to collect the data at in more continous 
duty cycle. Averaged data from this system appears to detect
the background atmosphere at ranges of approximately $4000 m$ 
and suggest that this maybe remedy for the poor detection 
capabilites of the present system \cite{evpd}.

Figure~{2.10} shows the possible techniques that can be used
to condition our present lidar backscatter profiles prior to inversion.
Here an original signal as collected is plotted. The value of the
most negative signal value is added to the entire profile along with
a small offset and is also plotted. Finally, signal values from
the topmost range height $(4000 m)$ down to $1500 m$ (the height 
at which we expect to begin to detect some of the background 
atmosphere) are averaged or fitted to 5 points above and below
the signal point of interest. Note there is little difference
between the fitted and averaged data result. In general it
is recomended that the data be averaged and not fitted. As the result is
mostly due to supposedly random noise it makes 
little intuitive sense to {\em fit} to it.

\subsection{Example of an Inversion of Real Data}

Figure~{2.11} is a plot of a range corrected signal (trace 6) from 
August 4th 1993 [93216]. It is shows relatively strong cloud return
around $1200 m$.

\begin{figure}
\vspace{5.0in}
\caption{Plot of a range corrected signal (trace 6) from 
August 4th 1993 [93216]}
\end{figure}

Figure~{2.12} shows several inversions of that signal using
a variety of values for the reference extinction coefficient,
with and without averaging. Regardless of the chosen boundary condition
or the averaging of the data, the inversion converges to 
a peak value of about $5{1\over{km}}$. Below the cloud layer,
from $500 m$ to $1000 m$ we see the boundary layer
aerosol contribution, which appears to be 
approximately $10^{-2}{1\over{km}}$. Values calculated for ranges
higher than $1500m$ are completely unreliable however.

\begin{figure}
\vspace{5.0in}
\caption{Inversions of a signal (trace 6) from 
August 4th 1993 [93216] using a variety of values 
for the reference extinction coefficient,
with and without averaging.}
\end{figure}

This cloud optical extinction result below $1500m$ compares well with
values predicted from models calculated by Wright et al.,
from a modified scattering model of Deirmendjian
\cite{rmm}.

