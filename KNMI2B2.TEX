\documentstyle [12pt] {thesis}
\input{macros}
%\input{psfig}

\begin{document}

\pagenumbering{roman}
\setcounter{page}{3}
\pagestyle {empty}
$\;$
\begin {center}
\vspace{2 cm}
{\Large\bf
Cloud Lidar Research at The Royal Netherlands Meteorological Institute
and KNMI2B2 Version 2 Cloud Lidar Analysis Software Documentation}
\vfill
{\large \bf Alexandre Y. Fong, M.Sc.}\\
\vfill
\vfill
{\large The Royal Netherlands Meteorological Institute (KNMI)}\\
\vfill
{\large Postbus 201,\\
	3730 AE DeBilt,\\
	The Netherlands}\\
{\large +33 30 206 911}
\vfill
\vfill
{\large October 1994} \\
\end {center}
\vspace {1 cm}
$\;$
\clearpage

%\DOUBLESPACE
\pagestyle {headings}
\begin {Large}
\begin {center}
{\bf Document Description}
\end {center}
\end {Large}
\vspace{1 cm}
\addcontentsline{toc}{chapter}{Document Description}

Clouds are without a doubt a critical piece in the climatological puzzle.
Specifically, the interaction between clouds and radiation form a
linchpin in fundamental understanding of both the
Earth's hydrolgical as well the radiative budgets. The Tropospheric
Energy and Water Budget Experiment (TEBEX) being performed by the
Atmospheric Research Section of the KNMI is motivated by this
interest. TEBEX combines an extensive set of observations from 
satellite, and passive and active ground-based sensors with input
from modellers. The goal is gain detailed insight into a 
$130{\times}130$km$^{2}$
area covered by 10 ground observation sites know as the Cloud Detection
System (CDS). Further details of the CDS program in its entirety are
given elsewhere \cite{psajfacapvlgjp}. 

The objective of the KNMI-TEBEX Lidar component are to obtain a set of cloud
optical and physical data as derived by lidar for comparison and analyses
with satellite data in a research aimed at exploring the Earths Radiation
Budget (ERB). To that end, software has been developed to 
process such lidar measurements to obtain the salient information.
This information includes: cloud base and top height, peak signal altitude,
physical thickness, optical extinction and depth. 

KNMI2B  is  a software package specifically created for 
the analysis of lidar (laser-radar) data obtained from the National 
Institute of Public Health and Environmental Protection (RIVM)
Laboratory for Air Quality Research aerosol lidar system in 
Bilthoven, The Netherlands for use by the 
Royal Netherlands Meteorological Institute (KNMI) in De Bilt, The Netherlands. 

This study was contracted as part of a scientific consulting contract on
cloud lidar studies by Dr. Andr\'{e} Van Lammeren of  The KNMI. The 
software was designed and authored by 
Mr. Alexandre Y. Fong as per Dr. Van 
Lammeren's requirements and specifications.
Ownership of this software is shared by Dr. Van Lammeren, Mr. Fong 
and The KNMI. These parties assume no liability for damage to devices
or existing software that may result from its subsequent use.

KNMI2B incorporates widely used algorithms and defintions to derive optical
and phyisical properties of interest and insight to those studying the 
effects and contributions of cloud to climate and atmosphere. The source
code is written in Microsoft FORTRAN 5.1 and incorporates an algorithm 
developed by the Institute for Space and Terrestrial Science (ISTS) in 
Toronto, Ontario, Canada by Dr.'s Allan I. Carswell, Shiv R. Pal, 
Wolfgang Steinbrecht.

\noindent
KNMI2B obtains the following parameters:\\

\begin{itemize}
\item Cloud Base Altitude
\item Cloud Peak Signal Return Altitude
\item Apparent Cloud Top
\item Mid-Cloud Altitude
\item Cloud Mean Optical Extinction [calculated from both the integrated
Klett inverted extinction profile and by comparing the above and below 
cloud background]
\item Optical Extinction (profile)
\item Optical Depth
\end{itemize}

KNMI2B2 also identifies and obtains cloud base height 
algorithm information but can, if the operator chooses to,
disregard inverting saturated profiles for optical data.
KNMI2B also provides output that has sonde information correlated to 
cloud base height optical and physical information listed above.

\clearpage

\begin {Large}
\begin {center}
{\bf Acknowledgements}
\end {center}
\end {Large}
\vspace{1 cm}
\addcontentsline{toc}{chapter}{Acknowledgements}

The author wishes to acknowledge the assistance and cooperation
of Eric Visser, Hans Bergwerth, Arnout Apituley and Daan Swart of the 
National Insititute of Public Health and Environmental Protections (RIVM),
Laboratory for Air Quality Research.

Special thanks to Dr. Andr\'{e} Van Lammeren and Carin, Joris, and Roos 
Van Lammeren for their kind hospitality and friendship. Special thanks
to Andr\'{e} for his insight and suggestions which kept the momentum
of this work going. 

Thanks also Pieternel Levelt and Wim Vassen for their friendship and
company, Arnout Feijt and Piet Stammes for all their advice
and techincal assistance with this project, Ronald 
Weber and Wieger Fransen for making
their home mine, Barbara Bongers, Sigrida Shperlina, Ad, Paul and Hanna, 
and making the bus rides to and from DeBilt shorter, 
Glenn (like a father to me!), Naci, Baelul, Sunil, Bibi, Lydia, Audrey, 
Everet, Ron Grovers (thanks for the postcard) and everyone else
at Gravenbeek's (the friendliest place in Holland!) 
and everyone else at the KNMI for making my stay a pleasant one.

Special thanks to the people of Optech Inc.: 
Adelhied Freitag, Dino Lenarduzzi, Taner Machupyan,
Victor Gorin, Martin Flood, Michel
Stanier, Larry Mitchel, Jim Bottoms, Rob Murenbeeld and Wayne Sziemetat 
for keeping me company and up to date on all the great gossip back home,
away from home. Thank you to Doug and Lorraine Houston at Optech and
Linda and Allan at the Harry Sherman Crowe Co-op
for also helping me keep all things residential under control back home.

Very special thanks to Stefan Mihalov, Michel Stanier, 
Adelhied Frietag, and the Steinbrechts
of M{\"{u}}nchen for all their kind hospitality in showing me their
special parts of Europe. Thanks to Victor Gorin, Bill Dykshoorn and 
Pam Dillon {\em (and friend)} for dropping by.

Thanks to all the other folks who sent me email and let
me share in their lives this last summer: Paul Fairlie, 
Dwaine Plaza, Denise, Jan McQuay, Sandra Scott and Dave Tatar.

As always thanks to Dr.'s Allan I. Carswell and Shiv R. Pal for their
support and wisdom throughout the years which made all this work
possible.

Kudos to all at the {\bf\sl Cirrus Technology Consulting} home office
in Toronto for a job well done.

Finally, thank you to as always my parents, for their love and support. 

\newpage
\noindent
``But trouble can come to nice places, too; trouble travels,
trouble visits. Trouble even takes holidays from the places
it thrives, from places like St. Cloud's.''\\


\noindent
``What is hardest to accept about the passage of time
is that the people who once mattered the most to us
are wrapped up in parenthesis.'',\\
John Irving, {\em The Cider House Rules}\\

\vspace{0.1in}  
\noindent
``Life is what happens when you're making other plans'',\\ 
John Lennon\\

\vspace{0.1in}  
\noindent
``...and I am oughta here!'',\\
Dennis Miller, Weekend Update: Saturday Night Live\\

\vspace{0.2in}  
\noindent
\copyright 1994, {\bf\sl Cirrus Technology Consulting Press} 
\clearpage

\baselineskip = 3.0ex

\addcontentsline{toc}{chapter}{Table of Contents}
\tableofcontents
\clearpage
\addcontentsline{toc}{chapter}{List of Tables}
\listoftables
\clearpage
\addcontentsline{toc}{chapter}{List of Figures}
\listoffigures
\clearpage

\pagenumbering{arabic}
%\DOUBLESPACE
\setcounter{page}{1}
\include{ch1}
\include{ch2}
\include{ch3}
\include{ch4}
\include{ch5}
\clearpage

\baselineskip = 3.0ex

\bibliographystyle{unsrt}
\bibliography{alex2}

\end{document}
