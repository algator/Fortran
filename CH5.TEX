\chapter{Conclusions}

\section{Diagnosis: Current Condition}

The fundamental aims of this work have been 
three fold:

\begin{itemize}
\item Establishment of quality and limits of data
\item Creation of a software environment for analysing
cloud lidar data
\item Insight into future research directions
\end{itemize}

\noindent
These goals have been met and will hopefully be the begining
of a continued effort to understand and characterize clouds
through the use of lidar technology.

As this software and documentation are in final
development, the data being collected from the RIVM truck-mounted aerosol
lidar is less than satisfactory. Among the problems include limited
range which excludes a vast majority of the clouds of interest and
places many of the tops of clouds recorded out of reach thereby making
inversion for optical extinction impossible. 

As the system is designed
for the purposes of boundary layer aerosol detection,  much of the cloud
data is over the recording range of the digitizer and is
``saturated'' numerically. That is, a flattening at the maxima of the
signal occours. In addition, a problem from saturation of the solid-state
detector has been observed. It is evident that high signals can induce
afterpulsing in the avalanche photodiodes resulting in artifact signals
above the cloud.

This also limits our ability to obtain reliable
optical characterstics of clouds. We currently identify and 
obtain cloud base height algorithm \cite{srpwsaic} 
information on all signals and can 
disregard inverting saturated profiles for optical data, 
albeit the fact that they comprise large proportions of 
our currently available data.

Some residual overlap signal at the near end of the range leads to
difficulties for the cloud base algorithm in locating the bottom
signal of very low clouds. This strong return appears as part of
the cloud signal hence the algorithm will identify it's increasing
values as the bottom and regard it as part of a very expansive
cloud. No satisfactory explanation has come from the RIVM regarding
this problem.

Negotiations with the RIVM have been underway since this project
was first proposed and included adopting a dual duty cycle
for the system allowing it to record data at two different
dynamic settings, one for clouds and the other for it's primary
function, recording aerosol returns.

This has been abandonned in favour of an alternate approach
utilizing an older aerosol system not currently in use.
The following recommendations were discussed and agreed upon:

\begin{itemize} 
\item System must be optimized-aligned for cloud-lidar measurements.
\item The data acquisition will be set to collect 500 bins of data at a 
vertical resolution of 20 metres, thus extending the systems range 
capacity to 10km.
\item Neutral density filter configuration will complement the dynamic 
range of system 1.
\item new data acquisition software must be implemented.
\end{itemize} 

After some delay these modifications were performed in time
for the NASA-LITE (Lidar In-space Technology Experiment) project.
Data has been collected in conjunction with the Space Shuttle
overpasses as hoped, analyses are pending.

\section{Future Plans: Prognosis}

Even with such improvements to our measurements, the SNR for
the molecular background is sufficently low that obtaining
a satisfactory range of optically invertable data with the 
available technology in the current manner will still be problematic.
A possible remedy is to take more continuous measurements than
the current 25 second and 4 minute rest duty cycle and averaging over longer
time periods, thus increasing effective exposure time of the 
systems detectors as previously discussed. The RIVM group has 
observed some evidence of the faint molecular signal at 
and beyond 4000 meters using this technique \cite{evpd}.

Proposals are currently pending approval to develop new systems in 
conjunction with the RIVM which will be better suited to obtaining
cloud optical properties. These include a system with a source
in the visible range, namely 532 nm. We expect such a system to have
better SNR for molecular, providing us with better boundary
conditions and hence more range in our optical inversions.

Of interest yet farther into the future, will be the possible
foray into lidar remote sensing of water vapor or atmospheric humidity.
The Royal Netherlands Meteorlogical Institute currently relies on a 
meteorogical tower equipped with wet and dry bulb thermometers for 
atmospheric moisture measurements at the Cabauw meteorological site. 
Recent advances in optical, laser and signal processing technology 
have indicated a potential role for lidar (laser radar) in the routine 
measurement of atmospheric moisture. 
Tables~{5.1-5.2} offers some comparisons between 
the two methodologies and together with some comments 
describing the advantages/disadvantages of either technology based on the
results produced by research systems currently in operation. 

\begin{table}
\caption{Comparison of Moisture Detection Capablitlites Between the KNMI
Cabauw Meteorogical Tower and a potential Raman Water Vapour Lidar}
\begin{center}
\begin{tabular}{|l||l|l|}
\hline
Specification &	Tower &	Raman Lidar \\
\hline
\hline
Range &	213$m$	& 3$km$ (solar-blind\\
&& \cite{drrc}) up to\\ 
&& 7 km (night-time\cite{dnwshmraf}\cite{shmdw})\\
\hline
Resolution & 20$m$ & 5--150$m$ (dependent\\ 
&& on processing modality) \\
\hline
Accuracy & in-situ measurement & within$^{+}_{-}10$\%  
when compared\\
&&to radiosonde data above 20\% \footnote{radiosonde 
measurements are unreliable for measurements of R.H. under 20}\\
\hline
Noise & N/A & 5--20\%(solar-blind, depends\\ 
and Uncertainty & &on haze), 
0.02\%(night-time)\\
\hline
Cost & unavaiable at this time\footnote{to upgrade 
to state-of-the-art technology} & 
approx. fl. 685.000,00\\ 
&&(\$ 500,000.00 CAN)\cite{bs}\\
\hline
Special & & potential ability\\ 
Features & & to measure other \\
&& atmospheric species
of interest,\\
&& {\em ie. clouds,aerosol $CO_{2}$},\\  
&& {\em etc.,for limited extra costs}.\\
\hline
\end{tabular}
\end{center}
\end{table}

\begin{table}
\caption{Comparison of Moisture Detection Capablitlites Between the KNMI
Cabauw Meteorogical Tower and a potential Raman Water Vapour Lidar,
continued.}
\begin{center}
\begin{tabular}{|l||l|l|}
\hline
Specification &	Tower &	Raman Lidar \\
\hline
\hline
Other & relatively routine,& 
currently only proven as\\
Considerations & low maintenance & a routine night time\\ 
& approximately  1 hour/week & 
operation system, daylight \\
&& system still limited mainly to \\ 
&& research measurements. Low \\ 
&& maintenence \\
&& during routine operation,\\ 
&& less than $1\over{2}$ hour\\
&& per week apart from data\\ 
&& download.\\ 
&& Will require periodically,\\ 
&& a skilled technician for \\ 
&& maintenance, mainly for \\ 
&& the laser component of \\ 
&& the system. Possibly \\
&& $1-2$ days every $2-3$\\ 
&& months, depending on the \\
&& duty cycle and laser\\ 
&& specifications.\\
\hline
Temperature & inoperable when freezing
& independent of outside \\
Limits & of wet bulbs occurs & ambient temperature \\
\hline
Safety & N/A & with marine radar \\
Requirements & & connected to the laser interlock,\\ 
&& intrabeam hazard is eliminated, \\
&& diffuse beam radiation is eyesafe.\\
\hline
\end{tabular}
\end{center}
\end{table}

Raman scattering is a weak molecular scattering process, whereby 
incident radiation is shifted by a fixed amount associated with 
rotational and/or vibrational-rotational transitions of the molecule of 
interest. The shift is as such particular to a molecule. It has been 
shown that the ratio between the Raman scattered energy from water 
vapor to nitrogen is proportional to the water vapor mixing ratio.

Raman vapor lidar utilizes these two factors in remotely obtaining 
water vapor profiles of the atmosphere. 

For water vapor detection, there are currently three lidar methods 
being used. Conventional Raman at 347 nm is restricted to night-time 
use, due to the intervening solar background that will saturate the 
detection system. Solar blind systems use the 266 nm line, and are 
operational during the daytime, but have limited range by comparison.
DIAL or DIfferential Absorption Lidar utilizes two absorption lines 
in the 720nm range and requires two incident wavelengths. By 
comparing the attenuation of one (on) to the other (off) it is possible 
to determine the concentration  of an atmospheric species. It has been 
successfully proven effective in measurements of atmospheric ozone but 
has been shown prone to limitations for water vapor. 

DIAL Water Vapour Lidar is competitve with Raman for tropospheric 
measurements but has reduced accuracy and precision with regards to 
boundary layer measurements. It also is prone to degradation in it's 
vertical range resolution due to narrowing and absorption line shifts
which result from decreasing pressure\cite{cwmrevwlwm}. 
Although such systems have 
proven superior to the Raman-based lidars for regions above the 
troposhere with low resolution requirements, it is clear the for ranges 
and levels of precision required by the KNMI, Raman is the most 
effective methodology.

In terms of long term technical direction, moving towards an 
advanced remote sensing methodology would be beneficial for the KNMI in 
order to maintain a competitive lead with regards to practical 
standards of meteorological measurements. This is already the direction 
undertaken for such routine measurements such as cloud ceiling height
\cite{psajfacapvlgjp}.

Schwiesow \cite{rls} has discussed the potential for a completely 
self-contained, portable meteorological tower 
with a range of 1 km as a 
motivation for the meteorological community to move towards utilizing and
helping to spur on the development of such technology. It is viewed 
that as the state of the art improves such a system will move from the
many research instruments currently in use to a commercially packaged 
group of laser-based remote sensing devices. These systems will 
measure all the parameters that currently rely upon in-situ devices 
mounted on an expensive and fixed location meteorological tower. 
Moreover all with greater precision and range. Lidar systems can 
currently profile atmospheric aerosols, temperature, wind direction and 
velocity, water vapor, cloud base height, apparent top height and 
ice-water phase content, ozone, $CO_{2}$ and a host of other atmospheric gases
and other residents of interest. 

At present, active remote sensing technologies can serve to 
complement a conventional meteorological tower array, but given the pace of 
growth in this field it is expected to marginalize the utility of 
the tower and eventually replace it.

\subsection{Other Improvements: Far-End Extinction Coefficient 
Inversions with Range Variable
Backscatter to Extinction Ratios}

In applying Eq.~{\ref{kfee}} to solve Eq.~{\ref{lideq}}, 
(the {\em Klett Inversion} \cite{jdk1}), 
it has been assumed neither the proportionality factor, $C$, 
in Eq.~{\ref{betasig2}} nor the exponent $k$ is constant. 

Several observational and theoretical studies have shown that
under a wide range of circumstances where backscatter from
aerosols dominate molecular ({\em ie. hazy, cloudy, foggy conditions
in the I.R.}), we may assume this relation is valid. Although it 
is possible formally, to account for the range dependence of $C$,  
this requires some independent knowledge of the spatial variation
in the atmosphere which is generally not available \cite{jdk2}.

If it is possible the obtain some functional dependence 
for the the consant $C$, such that,

\begin{equation}
\beta= B(r) \sigma^k
\label{BR}
\end{equation}

\noindent
where we have denoted the former constant in the relation 
$C$ as the range dependent variable $B(r)$. This range
range dependent variable then has a functional dependence
to ${\sigma(r)}$ which can be writtne in the form,

\begin{equation}
B(r)=f[{\sigma_{1}}(r)]
\label{brig}
\end{equation}

It is then it is possible to rewrite the {\em Klett Inversion} to
accomodate for this variable.

\begin{equation}
{\sigma} = {{{B_f}\over{B}}^{1\over{k}}e^{{1\over k}(S-S_f)}\over 
{[\sigma_f^{-1}+{2\over k}{\int^r {{B_f}\over{B}}^{1\over{k}}
e^{{1\over k}(S-S_f)}dr\prime ]}}} 
\label{bakvar}
\end{equation}

\noindent
given the reference value is known,

\begin{equation}
{B_{f}}=B(r_f)
\label{Bref}
\end{equation}

\noindent
We can utilize this improved solution by performing 
the basic {\em Klett Inversion} using Eq.~{\ref{kfee}}
and deriving and intial guess for the extinction
profile (denoted as ${{\sigma_1}(r)}$) in Eq.~{\ref{brig}}. 
This can be used to obtain values for $B(r)$  which can be used in the
new form of the inversion above, Eq.~{\ref{bakvar}}. 

This again is of course, only valid where the scatterers are
relatively homogeneous and still assumes the single scattering case only.

A proposal may allow for a study to examine the effectiveness
of this technique on real lidar data in the near future. This
would be a verification of this technique
using real data as compared to the original work which used data generated 
by other methods. Hence, it will be an extremely worthwhile
and enlightening project.
