%Documentation [2nd draft]
\chapter{Lidar Fundamentals}

% 1. lidar concepts and theory > backscatter and extinction

A typical atmospheric lidar system is comprised of two major
components: a transmitter, (ie. laser source), 
which sends out a light beam of controlled characteristics through 
the portion of the atmosphere of interest
and a receiver which detects the energy backscattered into 
it's field-of-view (FOV). 
This backscattered signal contains information on the scattering and absorption
processes taking place within 
that portion of atmospheric medium probed by the beam. If the lidar-atmosphere
interactions are understood then an inference can be made 
as to the composition or condition of the atmosphere.
With short pulse laser sources the time of flight of the return
pulse provides excellent range resolution. 

Two configurations of the transmitter-receiver are utilized
in lidar systems. The bistatic configuration places the transmitter 
and receiver at different locations.
% and is of particular interest
% for studies of depolarization of the incident radiation
% by the atmosphere\cite{aic1}.
All of our work has been performed on the more common 
monostatic configuration, where the transmitter 
and receiver are located together.
Our monostatic system has the transmitter and receiver axes aligned
collinearly. Fig.~{1.1} is a schematic of this configuration.
Monostatic lidar consequently relies upon measurement
of energy scattered at $180^{o}$ from the incident beam source.
By utilizing a laser source with short pulse lengths,
high repetition rates and large energies, and consequently
high average power, weak interaction phenomena 
at distant ranges can be examined with high spatial resolution
and good signal-to-noise ratios (SNRs).

\begin{figure}
\vspace{5.0in}
\caption{Schematic of the monostatic, biaxial lidar configuration
utilized. The laser and telescope comprise the transmitter and receiver,
$x$ is the overlap range.}
\end{figure}

The overlap range, $x$, is defined as the
range at which there is full overlap of the receiver FOV with the
transmitter beam.
Beyond this range the signal decreases inversely as the square of the
distance in a homogeneous atmosphere. At closer ranges 
the backscatter intensity increases rapidly
as the transmitter's beam falls within the receiver's
FOV. This intense region of backscatter
is often suppressed in many lidar configurations to
reduce demands on the detector's dynamic range.

For the lidar system we are concerned with, this portion of 
the range lasts for up about 800 m (5 us) and has a 
maximum signal of over 500 mV (the maximum range of the digitizer). A cloud 
layer signal (with the current optical attenuation settings on our receiver 
is about 50 mV). The noise level is about 1-2 mV.
 
Please note the two-way path involved in the time/distance discussion above.

\subsection{The Lidar Equation}

If only single scattering events are assumed to redirect the photons
to the receiver as they travel through their atmospheric path
then the well known lidar 
equation can be considered valid \cite{aic2}:
 
\begin{equation}
P(r)={P_0}{c\tau\over2}{A\beta (r)\over r^2}
e^{[-2\int_0^r{\sigma (r\prime)dr\prime}]} \label{lideq}
\end{equation}

\noindent
Where,
\noindent
\begin{tabbing}
$\tau$ \hspace{0.2in}\= = the pulse duration\\
$P(r)$ \>= the instantaneous received power from range $r$\\
\> at time $t$\\
$P_0$ \>= the transmitted power at time $t_0$\\
$c$ \>= the velocity of light\\
$A$ \>= the effective system receiver area\\
$r$ \>is the range $[={c(t-t_0)\over2}]$\\
$\beta(r)$  \>is a measure of the incident laser power backscattered\\
by the atmosphere per solid angle and per unit path length\\
and is known as the volume backscatter coefficient.\\
$\sigma(r)$ \> is the signal attenuation per unit length from the lidar beam\\ 
due to the scattering and absorption of the incident radiation\\ 
and is known as the attenuation or extinction coefficient.\\
\end{tabbing}
The nominal vertical (spatial) resolution of lidar $(\Delta z = c\tau/2)$
is determined by the length of the laser pulse.
The horizontal resolution is a function of the
beam divergence. Current laser systems are capable of producing
pulse durations on the order of $10$ns, as is the case for
our system. Furthermore beam divergences are typically on the
order of $1$mrad. This results in spatial resolution unattainable
by any other atmospheric remote sensing technique.

\subsection{The Volume Backscatter Coefficient}

The fractional amount of incident energy scattered per steradian 
back into the receiver per unit path length is 
given by the volume backscatter coefficient,
$\beta$. The $\beta$ has units of ${m^{-1}}{sr^{-1}}$
and is the result of scattering contributions 
from all atmospheric constituents. $\beta$ is often
expressed as,

\begin{equation}
\beta = {\sum_{i}}{\beta_{i}} = 
{\sum_{i}}{N_{i}} d\Sigma {(\pi)_{i}}/d\Omega 
% = \sigma_{s}{\frac{P_{\pi}}{4\pi}}
\label{backall}
\end{equation}

\noindent
for all $i$ atmospheric scattering species with number density $N_{i}$, 
and backscatter cross section $d\Sigma {(\pi)_{i}}/d\Omega$.
The latter varies in magnitude from the order of ${10^{-8}}{cm^{2}}{sr^{-1}}$
for aerosol particles to below ${10^{-28}}{cm^{2}}{sr^{-1}}$
for scattering due to Raman effects. The wavelength
dependence of the backscatter cross section also makes $\beta$ 
a wavelength dependent quantity \cite{agc}\cite{aic2}.
The $\beta$ may also be expressed as the product of the
volume scattering coefficient $\sigma_{s}$ and 
the backscatter scattering phase function $\frac{P_{\pi}}{4\pi}$

\begin{equation}
\beta = \sigma_{s}{\frac{P_{\pi}}{4\pi}}
\label{backall2}
\end{equation}
 
\subsection{The Volume Extinction Coefficient}

One of the most fundamental and important optical
parameters measured by lidar is the volume extinction coefficient
$\sigma$, defined as the extinction cross section
per unit volume with units of ${m^{-1}}$. 
In the integral in Eq.~{\ref{lideq}},
$\sigma$ represents a summation of all the atmospheric attenuation processes 
affecting the pulse in its two-way path through
the atmosphere. It is most commonly expressed as,

\begin{equation}
\sigma = {\sigma_{a}}+{\sigma_{m}}+{k_{a}}+{k_{m}}
\label{exall}
\end{equation}

\noindent
where ${\sigma_{a}}$ and ${\sigma_{m}}$ are extinction
coefficients due to scattering from atmospheric aerosol and 
molecular constituents, respectively. ${k_{a}}$ and ${k_{m}}$
are similarly the aerosol and molecular extinction
coefficients resulting from atmospheric absorption.
${k_{m}}$ is strongly wavelength dependent in comparison
to the other coefficients. At many wavelengths, strong
molecular absorption limits the propagation of
the transmitter signal and hence the utility of lidar.
Careful selection of incident frequencies within
the so-called {\em atmospheric windows}, where $k_{m}$ is
negligible, provides a remedy to this problem \cite{agc}\cite{aic2}.

For studies of clouds, $\sigma_{a} \gg \sigma_{m}$ and we need only consider 
a single term, $\sigma$, resulting from the scattering of the cloud 
particles. Related quantities
of cloud depth and mean volume 
extinction coefficient also are often useful.
In terms of the volume extinction
coefficient we define the optical depth of the cloud as, 

\begin{equation}
 \tau  = \int_{r_{b}}^{r_{t}} {\sigma}(r) dr 
\label{opdep}
\end{equation}

Given cloud base and top altitudes, 
(${r_{b}}$ and ${r_{t}}$, respectively), the 
{\em mean volume extinction coefficient} 
for the cloud may then be defined as, 

\begin{equation}
\overline{\sigma} = {\tau}/({r_{t}}-{r_{b}}) 
\label{meanex}
\end{equation}

\cite{cmrpjcsacd}. 
  
\subsection{Multiple Scattering}

Equation \ref{lideq} is less representative of backscattering 
in cases of higher turbidity where 
multiple scattering effects dominate. Such 
is the case for dense clouds and certain atmospheric aerosols.

Fig.~{1.2} is a very simplified illustration of the contributions from multiple
scattering to the backscatter in turbid media.
As the beam enters into the cloud the beam is broadened by
forward scattering due to the cloud droplets.
The three regions: A,B and C represent the continuation
of the original beam, the broadened beam still in the 
receiver's field FOV and outside the receiver's FOV,
respectively. Region A is predominantly the result of
single scattering whereas B and C are strictly due
to multiple scattering; the latter being the result
of at least the initial scattering incident and subsequent
events redirecting the photons back into the receiver's FOV.
In actuality, photons may undergo many more
scatterings within and outside the receivers FOV 
making the problem all the more complex \cite{aic1}. 

\begin{figure}
\vspace{4.0in}
\caption{Schematic representation of lidar beam
behavior in a turbid medium [32]}
% Re-check this reference number!!!!!!!!
\end{figure}

The degree of multiple scatter is a function of 
two main processes: the angular dependence of the scattering
and the rate at which multiple scattering occurs.
These are dictated by the medium's composition,
scatterer optical properties and number densities.
However, the measured contribution to multiple scattering
is also dependent upon the beam divergence and
the target medium's range from the lidar. Any multiple 
scattering lidar equation will necessarily be more complex 
than in Eq.~{\ref{lideq}} \cite{lrbdlh} \cite{cwjshhhgd}.
% Cite Bissonnette reference!

Efforts to include such effects remain an active field 
of inquiry. At present there exists no widely accepted modification of
Eq.~{\ref{lideq}} which adequately includes multiple scattering effects.
% input from Dan!
% Modelling techniques such as Monte Carlo simulations
% have been applied in an effort to reproduce actual measured effects.
Several radiative transfer model approximations as well
as time dependent multiple scattering models have been developed
\cite{kekjaw}\cite{jvdjg}\cite{dgcwgbmbwhgh}. 
The advent of low cost, 
high speed computing platforms have permitted some
advances but at present 
there exists no adequate atmospheric radiative
transfer models which fully account for multiple scattering 
in the lidar equation. 
The spatial ambiguity of the scattering volume
due to multiple scattering 
also degrades the time-dependent range resolution of lidar
since there is no longer a direct proportionality between the distance
to the scattering volume and the time of flight of the pulse.

%The main emphasis in this report is directed to relatively 
%dilute cirrus clouds in which multiple scattering contribution
%is negligible. Even in the denser, low clouds our receiver FOV
%is so small $(\sim 0.3\;mrad)$ that the contribution from multiple
%scattering will be too small to alter any conclusions of 
%the analyses undertaken in this report. 
%As it is not essential to the work covered in 
%this report, we will not elaborate further on multiple scattering. 
%
\subsection{Cloud Lidar Measurements}

Due to the large scattering cross-sections of the particles involved,
clouds induce large backscatter. 
As mentioned above the effects of multiple scattering
can introduce difficulties into the interpretation of
some lidar cloud {\em signatures}. 
By noting the lidar signal 
from the atmosphere above the cloud,
one can determine the level of signal 
penetration of the beam.
This aspect is particularly important to 
ensure that lidar has penetrated low-level 
dense clouds composed primarily of water droplets
\cite{agc}\cite{aic2}. However, there are 
many of these clouds for which the lidar
probing range is only one or two hundred metres.
Thus lidar data are limited by the ability of the beam to penetrate 
the cloud.

Lidar excels at defining cloud spatial distribution,
particularly cloud base altitudes and geometric thicknesses.
This has spawned the development of commercial ceilometer instruments
for the routine monitoring of cloud bases for use in aviation
\cite{wle}\nocite{wle2}--\cite{ned}. Although the inversion of 
lidar returns for the purpose of extracting atmospheric optical
parameters presents challenges, lidar offers measurement of such
quantities with unparalleled temporal and spatial resolution.
In clouds, the lidar scattering signature 
is an indicator of a cloud's interaction with solar 
radiation\cite{agc}\cite{drknl}. 
The combination of lidar and narrow beam radiometer
measurements {\em (LIRAD)} has shown much success
in furthering our knowledge of cloud
radiative properties \cite{cmrp1}
\nocite{cmrpacd2}\nocite{cmrph}\nocite{cmrpacd1}\nocite{cmrp2}
\nocite{cmrpjcsacd}--\cite{cmrpjdswdh}.
% - Think it's okay now. (Fix up this citation listing !!!!)

\section{Extinction Coefficient Inversions}

% Need a little intro material for that section
% saying what you are doing.

The reliable extraction of attenuation and backscatter 
parameters from lidar returns by inversion methods has been
a long standing and relatively unfulfilled goal of lidar researchers.
A number of factors, ranging from technological limitations
to theoretical constraints, have prevented the successful 
development of satisfactory inversion processes.
Particular interest has been given to obtaining
accurate values of the volume extinction coefficient
for reasons discussed in Sec.~{2.1.2}.

The {\em Inverse Problem} is central to all
optical remote sensing techniques. As is clear from Eq.~{\ref{lideq}},
the observable $P$ is dependent upon a number of
parameters. Most are system dependent and as such
can be determined in a relatively direct manner. 
The two atmospheric parameters, $\beta$ and $\sigma$, must be determined
by solving or ``inverting'' the equation.
% remaining unknowns are
% also the two quantities we are charged with the task of ascertaining. 
Both parameters are unique to the particular 
atmospheric optical situation we are attempting to
determine. This one-to-many mapping from
$P$ to these atmospheric parameters results
in an infinite number of combinations for the inversion solution.
Such an inverse problem is thus ill-posed or underdetermined.

To proceed, what is required are some constraints
to limit the possible solution set, and to
reduce the underdetermination of this problem.
In one approach, many measurements of $P$ are obtained 
with a controllable varying parameter such as
incident wavelength or beam angle. With the resultant simultaneous equations
% as to the wavelength or angular interdependence of our measurements 
the problem becomes solvable.
Such schemes often require complex measurement methods
% elaborate technological means 
and even so, are subject to the stringent limitations
of the assumptions utilized.
%  These assumptions are the product of idealized
% theoretical analyses and by no means definitive
% in the highly variable and non-ideal real atmosphere.

For inversions of monochromatic monostatic lidar
where one has only a single measurement of $P$, it
is necessary to add some additional information, usually in
the form of an assumed  
% is obtained, this issue becomes all the more
% pressing. In this case 
functional relationship between the unknowns,
$\beta$ and $\sigma$. 
% This section needs to be re-organized !!!! 
%\subsection
% When lidar eq'n is introduced make some comment about the problem 
% of 2 unknowns sigma, beta and need to add additional info such as
% beta/sigma ratio. Then refer to it here.
% As discussed in Sec.? , with respect to solutions of the lidar equations,
% In deriving such inversion algorithms a simplifying assumption, 
% between the relationship between the backscatter coefficient and the
% extinction coefficient is made. 
There is physical basis for 
this assumption, however in this report no attempt 
is made to describe this in detail. \cite{jdk1}.
% Although some further discussion is given in Sec.?.
%
% The optical parameters  $\beta$ and $\sigma$
% are dependent upon the microphysical scattering 
%properties of the atmospheric medium \cite{gdlgjkcwl}. 
% As previously discussed in Sec.~{2.1.2}, $\beta$ describes 
% the sum of all backscattering from all the aerosol species present at
% the time of measurement, for example.
%
% Was this done earlier?
%\begin{equation}
%	 \beta=\sum {n_{i}}\Omega_i 
%\label{beta}
%\end{equation}
% ...where $n(r)$ is the number density of the species $i$ and $\Omega_i$
% is the per particle backscattering cross-section of the species. 

% The number of factors which $\beta$ and $\sigma$ are dependent upon make it
% difficult to define a reliable relationship 
% for most atmospheric conditions 
% except for instances where one constituent dominates. 
% This is often true at high altitudes where
% only molecular components exist or in the lower troposphere where large
% aerosol cross-section contributions are orders of magnitude 
% larger than those of the molecular components. It is possible in such cases 
% to derive values such as the number densities of such components 
% \cite{aic1} from the lidar-derived $\beta$ values.
% This commentary would be better to come after eqn's 
% 2.25 and 2.26 gave been introduced!
%          ^^^^ has been deleted!
% Unfortunately, the myriad of processes involved in the formation and decay
% of natural aerosols pose additional difficulties in obtaining
% a valid relationship between backscatter and visibility 
% and subsequently extinction even under these conditions\cite{rwf}.
% 
% I think that details of values etc., would be better
% given as needed in the discussion of the results
% where beta and sigma are measured.

Conventionally, the isotropic backscatter-to-extinction 
ratio has been defined as \cite{cmrpjcsacd}:

\begin{equation}
C=\beta(\pi,z)/{\sigma(z)^{k}}
\label{betasig2}
\end{equation}

\noindent
where $\beta(\pi,z)$ is the volume backscatter
coefficient and $\sigma(z)$ is the volume extinction coefficient. 
Eq.~{\ref{betasig2}}, with $k=1$, is widely used to permit the solution
of Eq.~{\ref{lideq}} \cite{jdk1}.
% At $532nm$, Mie calculations are accurate for a given cloud drop
% size distribution for water clouds. $C$ is approximately $0.057sr^{-1}$.
% Generally clouds are given as 0.044 and dry aerosols as 0.024 \cite{wcrr}.
% For ice clouds there is uncertainty in such calculations for both 
% the extinction coefficient and $C$. From combined lidar and radiometer
% ({\em lirad}) observations, values for cirrus have been obtained at various
% altitudes and temperatures. For clouds below $-40^{o}C$, $C=0.013sr^{-1}$.
% \cite{cmrpjcsacd}.Fitzgerald, calculated backscatter to extinction
% coefficients as functions of the relative humidity fof the
% surrounding air for aerosols of varying chemical composition.
% has observed values ranging from $0.98$ to $1.28$
% for humidities between $90-88$\%, assuming dry aerosol characteristics
% do not vary along the path. 
% Generally $k$, the exponent in Eq.~{\ref{betasig}} 
% depends on the lidar wavelength
% and various properties of the aerosol. 
% Early workers reported the constant to be unity and $k=0.66$.
% Curcio, et al. Reported values are given
% as $0.67 \leq k \leq 1$ \cite{jdk1}, although some researchers have reported
% favorable results for $k=1.15$ using one-step inversions, 
% for cases where transitions between cloud
% and dust are not easily defined \cite{wcrr}.

Despite these considerations, an impressive
arsenal of approaches have been developed by 
workers. Although, in general, all these techniques have limitations, 
some can be applied with a large degree of confidence subject to specific 
instances. 
 
\subsection{The Slope Method Extinction Coefficient Inversion}

The simplest algorithm for the procurement of
extinction values from lidar measurements is the
so-called slope method. With reference to Eq.~{\ref{lideq}},
the lidar equation, we first define $S(r)$ as the 
'logarithmic range adjusted power'.

\begin{equation}
                 S(r)=ln[r^2P(r)]
\label{SR}
\end{equation}

% where $P(r)$ is the lidar return signal power. The use of this new variable
% compresses the dynamic range of the signal \cite{aic3}. 
% This is the convention in graphical
% analysis of extinction coefficients for the same reason.
\noindent
Rewriting Eq.~{\ref{lideq}} with this new variable it can now
be expressed as,

\begin{equation}
S-S_0=ln({\beta\over\beta_0})
-2\int_{r_0}^r\sigma dr\prime
\end{equation}

\noindent
Where $\beta_0=\beta(r_0)$.

The differential form of this equation is:

\begin{equation}
{dS\over dr}={1\over\beta}{d\beta\over dr}-2\sigma
\label{difform}
\end{equation}

As we have discussed, the solution 
of this equation requires knowledge of
the relationship between $\beta$ and $\sigma$
when $d\beta/dr \neq 0$. In the case
of a homogeneous atmosphere $d\beta/dr = 0$ and
${\sigma} = -{1\over2}{dS\over dr}$
Thus the attenuation coefficient
may be obtained from the slope of an $S(r)$ versus
$r$ plot.

By use of a straight line least squares fitting 
to the function $S(r)$, $dS\over dr$ can be
derived over small intervals over which 
the atmosphere can be considered to be approximately homogeneous with range. 
The slope method is therefore not considered a profile-
resolving technique, as the extinction values are actually
estimates of finite intervals.

This method can be applied under actual atmospheric
conditions in so far as an interval small enough 
to be truly said to be homogeneous can be achieved.
This is limited by the temporal and spatial
resolution of the system used. As such, it
also degrades this resolution by it's very
nature, one of the primary assets of the
lidar technique. Signal noise adds yet another
complication to this technique. To overcome
the effects of noise the chosen interval must be extended.
This again leads to a degradation of the spatial resolution
of the measurement.

This method also requires the restriction
${{\beta^{-1}}|d\beta/ dr|}\ll{2\sigma}$
to hold throughout most of $S(r)$.
Such rare quasi-homogeneous conditions
exist in a few dry climates, but
is unfortunately not the case
for most atmospheric conditions of interest.
It is particularly untrue of certain unstable and turbid media,
such as dense clouds, fog, smoke and dust.
Furthermore, in many stable less turbid media such 
as cirrus clouds, spatial variations lead
to sharp variances in $d\beta/dr$. Such
situations lead to an invalidation of this method
\cite{agc}\cite{jdk1}.

\subsection{Far-End Extinction Coefficient Inversions}

% Tie in intro here with intro material in section 2.4.1
% ie. want to find sigma and beta from Eq. 2.1, there are 
% several ways etc., etc....

A profile-resolved technique requires an ``exact'' analytical solution.
Proceeding from the previously discussed assumed relation given in 
Eq.~{\ref{betasig2}}.
%
% \begin{equation}
% \beta= C \sigma^k 
% \label{betasig}
% \end{equation}
% $C$:constant

\noindent
We can rewrite Eq.~{\ref{difform}} in a form
corresponding to the Bernoulli 
or homogeneous Ricatti equation:

\begin{equation}
{dS\over dr}={k\over\sigma}{d\sigma\over dr}-2\sigma
\label{diffeq}
\end{equation}

Which has the solution: 

\begin{equation}
{\sigma} = {e^{{1\over k}(S-S_0)}\over 
{[\sigma_0^{-1}-{2\over k}{\int^r e^{{1\over k}(S-S_0)}dr\prime ]}}}
\label{fareq1}
\end{equation}
% where $\sigma_0^{-1}=C$, our constant.

% PERSONAL NOTE: Maybe say more about near-end solutions! 

This is commonly known as the 'near-end' or 'forward inversion'.
This solution requires knowledge of $\sigma_{o}$ the boundary condition of the extinction
coefficient in the near field (close to the lidar). However, the difference 
of terms in the denominator leads to mathematical instability. 
The resulting solution is exceptionally prone to noise.
% even for relatively clear signals suggesting the requirement for an unattainable 
% level of accuracy in ${\sigma_0}$. 
These instabilities also propagate 
and grow as the inversion iterates with range.

One can modify this solution in order to avoid such difficulties using
an approach suggested by J.D.Klett \cite{jdk1}.
If we define $\sigma(r_f)=\sigma_f$, Eq.~{\ref{fareq1}}:

\begin{equation}
{\sigma} = {e^{{1\over k}(S-S_f)}\over 
{[\sigma_f^{-1}+{2\over k}{\int^r e^{{1\over k}(S-S_f)}dr\prime ]}}} 
\label{kfee}
\end{equation}

This `far-end' or reverse iteration inversion is often referred to as 
the `Klett Inversion'. The sum in the denominator results in 
significantly improved stability over the near-end solution. 
The form of the denominator also indicates
that the dependence of the solution on ${\sigma_f}$ decreases with increasing
range values. By far the most widely used inversions in lidar research
are numerical implementations of Eq.~{\ref{kfee}}. In our work
we have adopted this approach.

The above inversion can 
be recast by means of the identity ${\exp^{\ln x}}=x$ 
into the form,

\begin{equation}
{\sigma} 
= 
{{{P(r)} \times {r^{2}}}
\over 
{[
{
{{P(r_f)} \times {r_{f}^{2}}} 
\over 
\sigma_f
}
+{{2\over{k}} \int^r 
{
{P(r\prime)} \times {r\prime^{2}}
} 
dr\prime} 
]}} 
\label{wolfee}
\end{equation}

As suggested by Steinbrecht and Visser \cite{evpd}\cite{ws}. 
The advantage gained here, is 
that this inversion is defined for signal values of $\leq 0$. The
definition of the logarithmic range adjusted power,
Eq.~{\ref{SR}}, which is used in the Klett inversion, 
Eq.~{\ref{kfee}}, clearly does not permit this. 
Although Eq.~{\ref{kfee}} and Eq.~{\ref{wolfee}} are mathematically
identical, in our work it appears not to be the case
computationally from our experience with real and simulated data.
The author is unable to account for these discrepancies at this time.

Fortunately one can circumvent the problem of values $\leq 0$ in several ways
as we will discuss later.

\subsection{Direct Effective Mean Extinction Coefficient Inversions}

A value for the cloud optical thickness and subsequently
the mean value for the cloud extinction may be derived 
by comparing the difference in the background return signal from the 
surrounding atmosphere above and below the cloud.
% lidar before the incident pulse enters and 
% after it emerges from the cloud. 
This ratio of the background signal just below the cloud base and
just above cloud top is logarithmically proportional to the 
optical depth of the cloud. This requires the cloud be located in a 
relatively homogeneous region of the atmosphere
and thus the background scattering from above and below 
the cloud differs only by the two way attenuation by the cloud
of the above-cloud return. 

% \begin{equation}
% {P_{RA}}(r_2)={P_0}{c\tau\over2}{A{\beta_{RA}}(r)\over {r^{2}_{2}}}
% e^{[-2\int_0^{r_{2}}{({\sigma_{R}}(r\prime) + {\sigma_{A}}(r\prime)) dr\prime}]}
% \end{equation} 
% Say a bit more here and define parameters 
% - can refer back to equation numbers given earlier etc...
% \begin{equation}
% {P_{RAC}}(r_2)={P_0}{c\tau\over2}{A{\beta_{RA}}(r)\over {r^{2}_{2}}}
% e^{[-2(\int_0^{r_{2}}{({\sigma_{R}}(r\prime) + {\sigma_{A}}(r\prime)) dr\prime} 
% + \int_{r_{1}}^{r_{2}}{{\sigma_{C}}(r\prime) dr\prime]}
% \end{equation} 
% \begin{equation}
% {{P_{RA}}(r_2)\over {P_{RAC}}(r_2)} = 
% e^{-2[\int_{r_{1}}^{r_{2}}{({\sigma_{C}}(r\prime) dr\prime]}
% \end{equation}

The signal at the cloud base can be expressed as:

\begin{equation}
P(r_b) = {P_0}
{\frac{c\tau}{2}}
{\frac{A{\beta}(r_b)}{r^{2}_{b}}}
e^
{-2[\int_0^{r_{b}}
{{\sigma_{A}}
(r\prime) dr\prime}]
}
\label{basepower}
\end{equation}

\noindent
Where the parameters are as previously defined and with
the subscript $b$ denoting the atmosphere just below cloud
base. $\sigma_{A}$ is the mean attenuation of the atmosphere
below the cloud.
% 
% \noindent
% $\tau$ = the pulse duration\\
% $P_0$ = the transmitted power at time $t_0$\\
% $c$ = the velocity of light\\
% $A$ = the effective system receiver area\\
% 
% And here,
% 
% \noindent
% ${P_{b}}(r)$ = the instantaneous received power at time $t$
% from the cloud base\\
% $r_{b}$ is the range at cloud base\\
% ${\beta(r)_{b}}$ is the volume backscatter coefficient
% at the cloud base\\
% $\sigma(r)$  is volume extinction coefficient at the cloud base

\noindent
Similarly for the return from the atmosphere just above
cloud top:

\begin{equation}
P(r_t)
= {P_0}
{\frac{c\tau}{2}}
{\frac{A{\beta}(r_t)}{r^{2}_{t}}}
e^{-2[
{\int_{0}^{r_{b}}}{\sigma_{A}}(r\prime) dr\prime + 
{\int_{r_{b}}^{r_{t}}}{\sigma_{C}}(r\prime) dr\prime
]}
\label{toppower}
\end{equation}

\noindent
where, $\sigma{c}$ is the mean cloud extinction coefficient.
%
% \noindent
% ${P_{b}}(r)$ = the instantaneous received power at time $t$
% from the cloud top\\
% $r_{t}$ is the range at cloud top\\
% ${\beta(r)_{t}}$ is the volume backscatter coefficient
% at the cloud top\\
% $\sigma(r)$  is volume extinction coefficient at the cloud top

\noindent
Defining the scattering ratio:

\begin{equation}
R_{SC}
= {\frac{{P}(r_b)}{{P}(r_t)}} = 
\frac{{\beta_{b}}{r_{b}^{2}}}{{\beta_{t}}{r_{t}^{2}}}
e^{[2 
{\int_{r_{b}}^{r_{t}}}{\sigma_{C}}(r\prime) dr\prime
]}
\label{ratiopower}
\end{equation}

\noindent
We may then express the optical depth for the cloud as:

% You must say more to clarify the details here !

\begin{equation}
{\tau_{C}} = {\int_{r_{b}}^{r_{t}} \sigma_{C}(r\prime) dr\prime}
= {\frac{1}{2}} \ln (\frac
{{{P}(r_b)}
% {\beta_{t}}
{r_{t}^{2}}}
{{{P}(r_t)}
% {\beta_{b}}
{r_{b}^{2}}})
\label{deeppower}
\end{equation}

\noindent
Where we assume ${\beta_{t}}\sim{\beta_{b}}$ for a 
homogeneous surrounding atmosphere.

This method has the obvious advantage that it utilizes only
those returns that contain cloud signal and does not require
a profile of clear air either before or after the cloud
moves into the field of view in comparison to similar methods used
elsewhere \cite{anunstdwwcb}. This has obvious statistical
advantages in our confidence of the derived extinction.
In practical applications a Rayleigh background atmosphere
is assumed from temperature and pressure conditions obtained
from sonde launchings. 

Such a technique does not provide a
complete profile of cloud extinction
as is the case for 
% the previously mentioned
% solutions (slope method and 
far-end inversions.
%).
However, it does not require additional information
on the boundary condition and has been found to 
% given reasonably
% averaged values with a good SNR 
provide a reliable value for the mean cloud extinction.

\section{The Cloud Base Height}
% Intro on importance of cloud base height?
% a. intro on importance of cloud, base, top, etc.
% b. how it is presently done - limitations.
% c. what we did, advantages.
% d. what algorithms we used, etc. [base vs. bottom]
% e. need for averaging, importance. 
% ^^^^^^^^^^^^^^^^^^^^^ Not for cirrus - reference ?
% Too diffuse, discuss those items of main interest
% in the order of priority.
% DIMER!
% This lead up to your specific interests is too broad.
% Focus on items of importance.
% 1. What is present interest - GCM.
% 2. How is cloud base currently measured.
% 3. What are the problems.
% 4. Lidar improvements.
% 5. Correlation of "old" with lidar measurements.
% Our algorithm in brief (refer to paper).

Among the cloud parameters specified in GCM's, 
the altitude and thickness information are most
important. Cloud base altitudes 
relate directly to cloud formation and
frequency. 
% The rise of unsaturated convective currents and it's
% consequent temperature decreases at the dry adiabatic 
% lapse rate (${10^{o}}/1000m$) leads to an increase in the relative humidity
% of that air parcel. Cumulus clouds are generated when the air-mass'
% relative humidity approaches 100 percent. This altitude is known as the
% the lifting condensation level {\em (LCL)}, and also corresponds the
% cloud's base altitude \cite{jmmmdm}. 
%
% Cirrus and high altitude 
% clouds form due to by deep convective
% transport of moisture, synoptic scale lifting and the
% spreading of aircraft contrails (or condensation trails)
% \cite{wcrp}. Knowledge of cloud base leads to an understanding
% of the origin of cirrus from the updraft process as
% well as the contribution to cirrus cloud formation
% from the contrails, an anthropogenic source \cite{knl3}.
% Cirroform clouds can evolve
% from the tops of cumuliform clouds when an appropriate
% temperature drop occurs\cite{arlpvh}. 
% Thus cloud top altitudes give an 
% indication of the contributions from such processes also.
% Furthermore anomalous cloud formations, at say, elevated
% temperatures or lower than normally expected altitudes,
% can indicate the involvement of alternative mechanisms
% such as a presence of additional cloud condensation {\em (CCN)},
% or ice nuclei {\em (IN)}. 
Emission temperatures can be established given knowledge
of cloud base altitudes. Cloud elevation can provide insight into the 
distribution of atmospheric water vapor.
%  below, in and above
% cloud region. 
Infrared fluxes are modified by cloud droplet scattering and 
water vapor 
% {\em dimer} 
absorption.
% within the $8--13\mu$m atmospheric windows.
% This effect is particularly strong reducing cirrus cloud to
% ground atmospheric transmission up to 50 \%. 
Such information 
is essential to a better understanding of 
radiative processes in cloudy atmospheres.

Cloud base height\footnote{Although we have used cloud base height
and cloud base altitude interchangeably throughout this report,
the two quantities do differ slightly. 
The height of a point, e.g. the base or
top of a cloud, is the vertical distance from the
point of observation to the level of that point. The altitude
of a point is the vertical distance measured from mean sea level
to the level of that point\cite{WMO}. For our work, the differences are not
important.} is also an indicator of a cloud's role as an
atmospheric radiation feedback mechanism in the ``greenhouse effect''.  
For example, Ohring and Adler \cite{geh} 
have shown that a one kilometre increase in cloud
top altitude and subsequently vertical extent, 
can result in a $1.2^{o}\;K$ increase in surface
temperature. Identical increases in both base and
top or elevation, however, result in 
only a $0.6^{o}\;K$ in surface temperature.
% This suggests that the the greenhouse effect is
% inversely proportional to cloud base height.

\subsection{Definition of Cloud Base}
% Some further discussion here, bottom vs. peak

The definition of cloud base is highly dependent upon the
technique used to measure it.
Current methods for cloud height measurement include
the automatic rotating beam ceilometer (RBC), 
the automatic fixed beam ceilometer (FBC), ceiling balloons, 
ceiling lights, in-situ aircraft observations and of course 
lidar \cite{wle}. The instrument of choice 
for routine determination of cloud base height and cover has been 
the ceilometer. Ceilometers require a bistatic configuration. 
The rotating beam ceilometer (RBC) consists of a 
transmitter which is a vertically rotating modulated
lamp projection system and a receiver located below the cloud of
interest at a prescribed baseline distance from the transmitter. 
The return is recorded as function of the transmitter's angular displacement. 
Upon the transmitter beam's intersection with the cloud base
and the receiver's field of view, a maximum return is observed. 
A base altitude may then be determined by  triangulation. This is
the convention currently in use by the U.S. National Weather Service. 

RBC returns have been shown
to be greatly affected by cloud geometry and in-cloud multiple scatter
in comparison to lidar techniques. The less accurate but more compact 
and inexpensive fixed beam ceilometers (FBCs) reverse the
configuration by having the transmitter fixed at the cloud base and
scanning the receiver along the beam path, once lateral alignment
has been performed. FBCs offer a low cost supplementary cloud base
height retrieval technique to RBCs but are of comparatively lower 
precision. RBC's tend to indicate higher
base heights than lidars and aircraft pilot 
reports. The recent advent of low power, 
highly automated, eye-safe laser ceilometer
represent a cost-effective compromise between RBC's
and more expensive, 
% high maintenance experimental 
research lidar systems. Due to the
low energies of the commercial laser ceilometers,
they have lower SNR,
% transmitters in these systems,
% result in low SNR and thus 
poorer spatial resolution
and more limited range. 
% In fact, 
% commercial ceilometers tend to be limited in terms of range 
% limits between 30 - 2000 feet. 

Algorithms have been developed to estimate cloud base
heights from the LANDSAT Multispectral scanner
satellite radiance measurements.
Although such a platform would have the
advantage of global coverage, this technique 
relies on the shadow cast by the cloud 
and is thus restricted to certain 
types of small broken cumulus clouds
in specific orientations to the solar angle.
As such this technique is necessarily only
an estimation for a small set of clouds
types in highly idealized conditions.
\cite{tbsksrmwbawmn}
 
% Lidar systems are limited
% mostly by transmitter power, detector sensitivity and the signal to noise
% ratio. Present laser technology permits observations of
% mesospheric activity under night conditions. However, the expense and 
% fragility of most lidar systems make truly routine 
% measurements impractical at present. 

Lidars specify cloud base by either the altitude where
the rapid increase in backscatter signal begins 
(cloud bottom altitude, $r_{b}$) or the 
altitude at which the backscatter 
signal reaches its maximum (peak, $r_{p}$).
The peak signal will always be located above the
cloud bottom. The difference between the two altitudes
is dependent upon the scattering properties and hence the
microphysics of the cloud in question. This difference can
be substantial in some cases but generally
becomes less significant as the altitude of the clouds dealt with increase. 

The choice of which definition to use for ``cloud 
base'' is somewhat dependent upon where the information 
will be employed. Aviation visibility applications 
for example may favor the bottom signal location and where cloud opacity begins
whereas for GCM implementation cloud extent 
is critical and the peak cloud signal location may be better. We will present
further considerations and discussions for selecting between these
definitions later in this report. 
% In the interests of our work in ECLIPS and here, we have selected
% the {\em cloud bottom altitude} as our value for discrimination where
% the cloud begins. 

Lidar systems allow for measurements of cloud
positions with unparalleled precision and improved accuracy 
\cite{knl3}\cite{drdlfr}. Other cloud macroscopic
parameters such as cloud top and consequently cloud thickness 
can also be obtained from lidar profiles. However, 
this precision can contribute to some of the difficulties
involved in deriving cloud altitude information.
The complex structure of clouds are 
highly resolved by lidar. This leads to signatures
with multiple peaks and high variances in even single cloud
layers. As discussed, multiple scattering effects
also add to the ambiguity in the measurement cloud position.
In contrast, RBC's yield rather low-featured, smooth
profiles from which cloud base altitudes are easily
distinguished, albeit with comparatively 
limited resolution and range.

\subsection{The Cloud Base Height Algorithm}

Although it is possible to obtain cloud altitude information
from direct assessment of individual lidar signatures,
evaluating large 
% ECLIPS 
data sets required automation
of this process. Subsequently, an algorithm, developed by members 
of the ISTS-York University Lidar group,
to extract such information \cite{srpwsaic}, has been made use
of here to a great extent in this study. 
It is one of two algorithms used in the 
ECLIPS International effort to extract such information. 

The algorithm utilizes the zero-crossing or sign change
of the first derivative of the lidar return signal power $dP/dR$ to
identify the rapid increase in return signal which results from the larger cloud
droplets present in the cloud. 
Many such crossings occur in practice resulting
from inhomogeneities in cloud structures, as well as from signal noise.
It is therefore necessary to develop a set of criteria, to 
filter out such spurious results. 

$dP/dr$ is determined from a least squares fit
of a sliding window of successive points
from the start to  end of the profile. 
A  $3$ to $5$ point fit has been determined to be suitable for
our data. By combining higher resolution in the range
with a larger number of points for the least squares fit, better smoothing
of the $dP/dr$ curve and a minimization of spurious zero-crossings, resulting 
from noise, may be obtained. It is noted, that for lidar signatures 
containing sharp changes in curvature, an upward or downward bias, 
depending on the sign of the curvature, 
in the altitude of the zero-crossing is introduced.  

% L^

A noise level is calculated corresponding to the return
signal at each cloud base ($r_{b}$) 
and a peak altitude ($r_{p}$) identified.
Zero-crossings where the difference between the return
at the base and adjacent peak is less than twice the noise
calculated at the peak altitude are regarded as insignificant 
and rejected. This criteria is also {\em scale-able} using a multiplicative
factor ({\em corrfak}) and is dependent on the system data acquisition
characteristics (ie. detector and electronic line noise) and 
contributions from optical modifications of the backscatter 
received by the detectors (ie. {\em ND} or neutral density filters).

The current rejection criteria for peak signals is given by the expression:

$$
P(r_{base})-P(r_{peak}) \leq corrfak \times 
\sqrt{P(r_{peak})}\sqrt{87 \times {range
\: resolution} \times {10^{ND}} \times 1.16}
$$

The constants in this rejection criteria formula were 
calculated for the ISTS Nd:Yag Lidar system in Toronto, 
Canada. The technical details required for an identical 
calculation to be performed on the RIVM system are
unavailable at this time. We have however obtained 
satisfactory results with the highest possible
sensitivity ({\em corrfak}) setting (0).

Apart from lidar and in-situ aircraft observations, 
existing ground-based cloud monitoring systems 
are incapable of measuring cloud top altitudes.
It is possible in all cases for lidar to extract
an {\em apparent} cloud top altitude when the signal
falls into the noise. If optical attenuation is not too high 
the lidar pulse penetrates the cloud completely and 
the signal transition from the cloud top to the clear air 
identifies the true cloud top height.

The cloud top altitude, $r_{t}$, is defined in the algorithm as the point just below 
which the return from the cloud decays to the value less than or 
equal to the backscatter signal at the cloud base. 
For cases where no measurable clear air backscatter exists at the
base, the cloud top is located where the signal drops to the same
background intensity as below the cloud. Where the sub-cloud backscatter is
well defined, the top is retrieved with improved accuracy by comparing
the range-corrected signals.

\begin{equation}
{r_{t}^{2}}{P(r_{t})}{ \leq}{r_{b}^{2}}{P(r_{b})}
\label{rangecorr}
\end{equation} 

The algorithm is capable of locating cloud parameters 
for multi-layer structures. In such a case, it is
given that a region of clear air must exist between  
adjacent layers. For each $r_{b}$ value, 
corresponding $r_{t}$ values which drop to the clear
air value are located. All other peak returns between
each base-top pair are ignored. The procedure is
repeated for all layers.

%Although the system is capable of sub-second temporal
%resolution, such resolution is neither required 
%nor prudent. Such return profiles
%have inherently low SNR. Averaging
%provides a more accurate representation.
%Since, in any time series, cloud layers begin and
%end at random intervals, attempts to derive time averages for
%$r_{t}$, $r_{p}$ and $r_{b}$ must take this into account, otherwise
%such averages would become meaningless values derived from 
%a superposition of multi-layer data. 
%The ECLIPS format
%incorporates a ten-minute time interval which is automatically
%started and stopped at cloud boundaries as necessary.

%In such an averaging, a stipulation is made that,
%only points not too far displaced spatially or temporally
%are joined. An initial value for a vertical window 
%is chosen ($_{-}^{+}{500m}$ or less for low clouds),
%and all adjacent points in this window are joined. 
%Vertical jumps greater than this are assumed to be cloud 
%boundaries. If an adjacent point is not available, either
%due to our criteria or from signal drop out, the next point
%is examined with a lower allowed vertical displacement value.
Fig.~{1.3} is a schematic of the cloud base algorithm.

\begin{figure}
\vspace{5.0in}
\caption{Schematic of cloud base algorithm}
\end{figure}

Horizontal extent of the clouds can be determined using the cloud base algorithm
at varying sensitivities to optimize 
location of the peak returns as a function of time.

\section{Data Averaging Techniques}

% too brief
% state problems encountered in deriving "average" cloud properties
% Not clear.
% give careful detail !
% Say more here to make procedure clearer!
% Should rewrite whole section on the normal and enhanced
% method to make things clearer.

The complexity and high sensitivity of lidar systems 
make their detection components prone to spurious contributions
from a number of sources including 
dark current and background noise.
Clouds persistent in time properties allow for
an increase in the SNR proportional
to the square root of the independent number of samples averaged.

The practice of averaging lidar profiles
came with the inception of the earliest lidar systems.
The higher the resolution of a system the fewer the photon events
one can expect to be detected per trace. 
Noise becomes a significant contribution, 
specifically in higher altitudes 
where photon counts are typically low. 
The averaging of lidar returns requires that 
such measurements must be made within rapid succession,
such that the atmospheric conditions have not varied
substantially within the elapsed interval between such measurements.
This requires that the backscattering characteristics be on
the same order of magnitude as other sources of noise
\cite{mjtmptw}\cite{nmdkkcrm}. Essentially, averaging
extends the effective {\em ``exposure time ''} of the detectors,
but without the degradation of pulse length and hence resolution 
and problems of saturation.

Furthermore, for the most part, clouds are
extremely inhomogeneous. Single lidar profiles of clouds alone
are rarely representative of the cloud as a whole
and as such are not very informative. 
For studies of radiative transfer, meaningful characteristics
from inhomogeneous clouds are often only apparent from data
that has undergone substantial averaging processes. 

However, within a lidar 
time series, clouds often move vertically
as well as horizontally, across the field of view. 
Since clouds are inhomogeneous, 
simply averaging the data over range intervals
as previously described inevitably leads to mixing or
contamination of the cloud portions of the signal with
non-cloud regions. If this were done the cloud's 
opacity would be diluted by the ``averaging in'' 
of clear atmosphere. This would lead to an underestimation
of the clouds actual optical depth. Conversely averaging cloudy regions with
clear or non-cloud portions of the profile would lead to 
% contain some cloud signal leading the enhanced backscatter
an overestimation of the clear atmosphere's optical extinction.
Hence the averaging of subsequent lidar profiles in time
must be approached with attention to not compromising the optical
characteristics of either the cloud or background atmosphere.

It is also possible to average lidar signals with respect to
range. That is, to attempt to minimize the effects of noise
by summing and averaging consecutive points along the lidar
range resolved profile. This degrades the vertical
resolution of the measurements involved, and again care must
be taken not compromise the cloud optical presence. However,
given a sufficient vertical range, and avoiding the cloud portions
of the trace, this can be quite an effective treatment to noisy
background signals especially in the higher ranges where the
SNR is low.
