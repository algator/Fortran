\chapter{Other Software}
\section{Lidar Simulation Software Set}

To test the performance of the inversion algorithms used in our
analyses, a set of smaller programs were developed to generate
simulated extinction, lidar power and inverted profiles.
These programs will be included in the CirrusWare directory under
the subdirectory SIMU. All source codes have also been included.

RAYGEN3 generates a Rayliegh pure molecular atmospheric extinction 
profile either by assuming a lapse rate of ${-6.75^{o}}{C/km}$ and 
the hydrostatic pressure equation or from a given radiosonde profile.
Cloud layers can be inserted as desired each with a user specified
optical extinction value.

RAYGEN3 requires input parameters to be read from a file
known as master2.dat which has the following format:


\noindent
line 1: name of output file.\\
line 2: lapse rate or sonde generated profile for sonde (1).\\
line 3: the number of points in the sonde data.\\ 
line 4: the name of sonde data file.\\ 
line 5: how many points to generate the profile from 
        using the lapse-rate/hydrostatic equation.\\
line 6: what resolution (metres) to generate the profile from 
        using the lapse-rate/hydrostatic equation.\\
line 7: add noise (1) or skip (0).\\
line 8: desired signal-to-noise ratio.\\
line 9: how many cloud layers to add.\\
line 10-n cloud layers: mid-cloud height simulated cloud in metres, maximum
                        extinction value at peak (1/km), thickness in metres\\


\noindent
example:\\
\noindent
{\tt bigclde.dat\\
0\\
15\\
9321600.dat\\
100\\
40\\
0\\
1.0e+20\\
3\\
3000.,1.0,200.\\
2000.,0.3,80.\\
1000.,0.2,320\\
}


\noindent
The ouput has the following format:

\begin{center}
		{\tt 1x,h,1x,s,1x,b,1x,k}
\end{center}
\noindent
where,

\noindent
nx: n character space(s).\\
h: height in metres in integer 5 digit wide if generated from sonde,
   real 7 digit wide, 0 decimal place format otherwise.\\
s: optical extinction in 1/km as real 10 digit wide, 3 decimal place format.\\
b: backscatter coefficient in km as real 10 digit wide, 3 decimal place format.\\
k: backscatter to extinction coefficient ratio in real
   10 digit wide, 3 decimal place format.\\

PGEN2 is an executable that generates a lidar backscattered power
profile (or return) from a given extinction profile.
Upon starting the program, it will ask the user to provide
the following input parameters (typical values given):


\noindent
{\tt
extinction profile file name: 	1cloud.dat\\
ouput filename:			p1bcloud.dat\\
number of data points:		100\\
range resolution:		40.\\
starting value:			1.0e-6\\
multiplicative factor,\\
represents the geometric\\ 
and optical configuration\\
of the hypothetical lidar system\\
being simulated:		1.0e-6\\
}

\noindent
It reads extinction profiles formatted identically as the output of
RAYGEN3 and write to the specified file the following output:

\begin{center}
		{\tt 1x,range,1x,backscatter power}
\end{center}
\noindent
where,

\noindent
nx: n character space(s)\\
range: height in metres real 7 digit wide, 0 decimal place format.\\
backscatter power: as real 10 digit wide, 3 decimal place format.\\

\noindent
These values can be entered directly through the keyboard,
following the on screen prompts or through a redirect of an
input file.

REVERT2 will invert the single backscattered profile generated by
PGEN2 using the techniques utilized by KNMI2B2.FOR.
It requires the following input parameters, example filenames
are given here:


\noindent
{\tt name of file containing\\ 
simulated lidar profile\\
generated by PGEN2:	 	p1bcloud.dat\\
name of file containing\\
simulated extinction profile\\
generated by RAYGEN3:		 1cloud.dat\\
name of output file that\\
will contain cloud base height\\
and mean optical extinction and\\
optical depth information:	det.dat\\
name of output file that\\
will contain the extinction profile\\
from the inverted data from\\
PGEN2:				p1bclex.dat\\
}


\noindent
It also requires a file known as para2.dat: 


\noindent
{\tt
npts,mpt,dpt,tpt,inpts\\
100,3,3,0,0\\
rez,zero,nob,fax,high,nlog\\
40.00,13.6,1,1.0,1.0e+20,1\\
corrfak,xovlp,ieb,k,setref\\
0.0,0.0,401,1.0,0\\
nd,gain,nos,cban,baseline2\\
0.0,0.0,250,1,1480.0\\
thrush,ntot,iabove,iwk,baseline,add,plusig\\
100.,14,0,0,1,0.0,0\\
}

\noindent
This file is almost identical to para.dat used for
KNMI2B2, but with the following differences:


\noindent
isat: is not specified or read here.\\
fax: is a multiplicative factor used to increase
the absolute value of the signal so that slope changes can be
easliy detected by the cloud base algorithm. It is identical 
to {\em multi} in the para.dat file for KNMI2B2. It can be set 
to any number that the user wishes to mutiply the entire profile by. 
This is useful extremely small values of signal where computational
limits may interfere with inversions or cloud basing routines. 
Since all factors are divided out in the inversions, this value 
has no effect on our optical derived qunatities [real].\\
cban: selects whether or not to use (0) or skip (1) 
the cloud base algorithm. If skip is selected then the program
defaults to setting the reference inversion height at npts, or
the highest range index, which is what is done if no clouds
are detected [integer].\\
nlog: selects between using the logarithm (1) or linear (0)
backscatter signal in the cloud base algorithm. For 1064 (I.R.)
data use the logarithm [integer].\\
baseline2: defaults all values in the backscatter profile above the height 
it is set to to a constant value specified by {\em zero}.  
Unlike {\em baseline} these value are set to {\em zero} regardless
whether or not they are greater than or less than [real].\\
setref: sets only the reference height signal point to the 
{\em zero} when it is set to 1 [integer].\\

\noindent
The remaining parameters not listed here have the same function as and
are described in the previous chapter in Section~{3.1}.

\noindent
REVERT2 produces an output in the format, for example, in the file det.dat:

\begin{center}
	{\tt cbh1,ph1,ath1,cbh2,ph2,ath2,cbh3,ph3,ath3\\
	nl,op1,op2,op3,me1,me2,me3}
\end{center}

\noindent
cbhn: cloud base height for layer n, real 7 digit wide, 0 decimal place format\\
phn: peak return height for layer n, real 7 digit wide, 0 decimal place format\\
athn: apparent top height for layer n, real 7 digit wide, 
      0 decimal place format\\ 
nl: number of cloud layers detected in a 2 digit wide integer.\\
opn: optical depth for cloud layer n in exponent 
     10 digit wide, 3 decimal place format.\\
men: mean optical extinction in 1/km for cloud layer n in exponent 
     10 digit wide, 3 decimal place format.\\

\noindent
The file p1bclex.dat, contains a single extinction profile
in columnar format given as:

\begin{center}
	{\tt rng,sig,sigs,sigp,logsgnl}
\end{center}

\noindent
rng: range or height in metres in a real number 7 digit wide, no decimals.\\
sig: optical extinction in 1/km in exponent 
     10 digit wide, 3 decimal place format derived using the Klett Inversion.\\
sigs: optical extinction in 1/km in exponent 
     10 digit wide, 3 decimal place format derived using a 
      modified Klett Inversion.\\
sigs: optical extinction in 1/km in exponent 
     10 digit wide, 3 decimal place format derived using a 
     Slope Inversion.\\
logsgnl: log of the backscattered siganl in exponent 
         10 digit wide, 3 decimal place format.\\

\section{Miscellaneous Support Software Included:}

\noindent
These programs can be found in the subdirectory MISC

\noindent
REPROC	Prior to August 1st, 1993, data provided by RIVM
        was given in the following format:

\noindent
\begin{center}{\tt	
	Mail header (17 lines) [before April 12, 1993]\\
	date \\

	...followed by the data set in the format:\\

	time stamp\\
	10x12 data points, 120 points of which only the first
	115 are valid]}
\end{center}

Once launched, REPROC prompts the operator for the number of files
to process, the header length as described above and the name(s) 
of the lidar file(s) to reformat and the name(s) to write the 
reformatted data to, in pairs. This can be done manually through
the keyboard, or by redirecting an input file such as through a
batch job.

\noindent
Once complete, REPROC reports:

\noindent
\begin{center}
	$>$DONE!
\end{center}
\noindent
to the operator

KEXTRA extracts individual traces from time series data files 
which are in the format of RIVM ASCII data or the optical extinction 
profiles generated from KNMI2B2 where each return profile 
or trace is written as a 100 data point row.
When launched it prompts the operator for:


\noindent
\begin{itemize}
\item name of file to read the data from.
\item name of file to write the single trace output to.
\item number of points in the data (prompted as {\em $>$npts}).
\item range resolution and the most lowest value the data
can equal(prompted as {\em $>$rez,zero}). 
If values are less than this value ({\em zero})
then they are set to it.
\item total number of traces in the file
\item number of the single trace to extract from that file.
\item Is the file RIVM ASCII backscatter (1) or KNMI2B2 extinction,
non-range corrected or processed averaged (0) data format?
\end{itemize}

\noindent
These values can be entered directly through the keyboard,
following the on screen prompts or through a redirect of an
input file. Kextra writes them  to single profile file in the format:

\begin{center}
	{\tt range, signal or extinction, range corrected signal or extinction}
\end{center}

\noindent
range: specified as a real seven digit, 2 decimal point 
value.\\
signal or extinction: data points are 10 digits wide, 
real with 3 decimal places and a single space is placed between columns.\\

CELTRAK this program extracts indivdual or groups of profiles from
ceilometer data collected at Cabauw. The data normally has the format of
107 data points per row as 2 integer digit wide ASCII data seperated 
by single spaces between points. CELTRAK also requires the input file 
inst.dat which has the format show below:


\noindent
{\tt
93216348.DAT\\
93216348.CAT\\
107,401\\
1,312,0\\
}


The first line specifies the ceilometer file from which the data
is to be read. The second line specifies the file to write to.
The third and fourth lines have the following format:

\noindent
npts,ieb\\
istart,notra,sw\\

\noindent
where

\noindent
npts: are the number of points in the data per profile.\\ 
ieb: the total number of profiles in the file.\\
istart: profile at which to begin extracting data.\\
notra: number of subsequent traces to extract.\\
sw: option to write out data in a single column (1) [only useful for
    single profiles] or rows (0).\\

CELTRAK outputs the data in the format of a 107 data point row in 
2 integer digit wide ASCII data seperated by single spaces between
points or for the case of a single profile, in a 3 integer digit
wide single column, depending upon the selection of {\em sw} by the user.

\section{Software for Catalyst:}

ALEX4 this exectuable converts the stored in Catalyst formatted binary 
files to ASCII in either columnar or row (as described for KNMI2B2)
formats. Upon execution, it prompts the operator to:

\begin{center}
	$>$ENTER NAME OF FILE LIST
\end{center}
\noindent
and example of a file list, dat.lst, is given below:

\noindent
{\tt 4.7,1,0,0,1\\
ALLHIGH.OUT\\
1000HIGH.001\\
1000Hout.001\\
1}

\noindent
From left to right the five values seperated by commas 
on the first line are:

\begin{itemize}
\item range resolution in metres (4.7 in this case).
\item number of signals to average over (1).
\item number of indices or length of pretrigger to be used in generating
\item a background to be subtracted from the rest of the actual signal
(0 skips background subtraction).
\item a switch that selects whether or not to range correct the data
(1=range correct, 0=no range correction).
\item another switch which determines whether to write the data in
rows (0) or columnar data (1, only valid for single trace extraction).
\end{itemize}


\noindent
If column format is selected then the profile is written in a single
column from bottom to top, in 7 digit wide integer ASCII data format. 
Otherwise  each return profile or trace is writtten as a 100 data point row 
6 digit integer wide ASCII data.


\noindent
The second line specifies the name of the file you wish to write the
average of/or all the individual profiles from files you will look at.


\noindent
Two filenames are and a the block number to be read are 
specified in groups of three or from the third to fifth lines
for a single data file.


\noindent
The third or first line of this cycle specifies the 
name of the file you wish to read.
The fourth or second line of this cycle specifies the 
filename you wish to write the data to if you are writing single
traces from single trace files to single trace files.
The fifth or third line of this cycle gives the number
of the block to read (this is almost always 1).
