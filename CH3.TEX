\chapter{KNMI2B2 Main Code General Requirements:}

KNMI2B2 was written and designed to operate on any PC Compatible
platform. All graphical examples have been generated though
the software package Origin by MicroCal Software Inc. in a Windows 3.0
environment. The source code is written in MS-FORTRAN 5.1 and is 
approximately 1900 lines in length. Currently there are 32 adjustable
parameters which will be described In this chapter.

\subsection{KNMI2B2 Main Code Operating Description:}

In our description and instructions we will be using the sample data 
set 93216riv.dat provided to us by RIVM. This file is formatted with 
each return profile or trace as a 100 data point row 
6 integer digit wide ASCII data seperated by single spaces between points.
KNMI2B2 reads this data in as real numbers with a width of 7 digits and no
decimals. In the past, the data has been range corrected, although
the user can specify uncorrected data also, provided the data is
similarly formatted. As each trace is read a saturation flag in a
time index file is also read. The flag indicates if a signal contains 
saturated data.

The parameter file known as para.dat which gives the program it's
operating instructions for performing calculations, the lidar
data and knmi2b.lis which assigns filenames to the many program 
data products are read first. Data such as lidar radiosonde filenames
are also read here. The lidar data is read in the format
described above. Range correction if required, 
is removed and an array is created with the logarithmic 
values of the signal for use in the cloud base
algortihm. At this point the user can have specified in the para.dat
file to multiply or sum to all the data by a constant, set all data
below a certain set value to that set value and/or smooth the data
by averaging or fitting a line from the topmost range height 
down to any specified height.

Data and parameters are then passed to a subroutine
called club which locates the cloud base, peak return signal and
apparent top or penetration height index from the array containing
either the logarithmic or linear values of the lidar backscatter
(depending on the users discretion). For $1064 nm$ a logarithmic 
signal is preferred. This is a modified
version of the ISTS cloud base height algorithm known as icb.for.
It is currently limited to locating up to 3 cloud layers \cite{srpwsaic}.

If the data is not saturated then the radiosonde 
data is read in next. From it, a Rayleigh pure molecular
atmosphere is calculated and a value for the boundary condition or
reference extinction coefficient is interpolated beween the two 
points bounding the inversion reference height in the sonde data.
The reference height can be specified either as some fixed height
above all clouds in the time series or at a fixed set height in para.dat. 
The user can choose from the far-end or Klett, a 
modified Klett for $k=1$ described in the previous
chapter by the Eq.~{\ref{wolfee}}, or slope method inversion 
algorithm to invert the data and 
obtain an optical extinction profile \cite{jdk1}.

A subroutine then calculates the optical depth and mean
extinction from this profile. These two data are also calculated
by comparing the signal just above and below the cloud, using the
{\em ratio method}. Finally the meteorological parameters read in from the
sonde data are correlated with the clouds optical and geometric
properties and all of these results written to the previously specified 
files listed in knmk2b.lis.

If however the saturation flag is positive and the operator chooses to, 
the previous steps following the cloud base height algorithm can be omitted.
All files are then closed and the program terminates.

\begin{figure}
\vspace{8.0in}
\caption{Flow chart of the KNMI2B2 software features and 
data processing scheme, part 1.} 
\end{figure}

\begin{figure}
\vspace{8.0in}
\caption{Flow chart of the KNMI2B2 software features and 
data processing scheme, part 2.} 
\end{figure}

\begin{figure}
\vspace{8.0in}
\caption{Flow chart of the KNMI2B2 software features and 
data processing scheme, part 3.} 
\end{figure}

This is a completely self-contained program. There are no outside
objects to link as all the subroutines are attached to the main source
code listing, should modifications and a recompilation of the program be
required. Below is a flow chart (based on ANSI X3.5-1970) 
of the program schematic, it will be useful
to refer to this when using this package (see Figures~{3.1-3.3}).

\section{KNMI2B2 Operating Requirements: Support Files,Inputs and Formats}

KNMI2B2 relys on four input files for it's operation: para.dat, knmi2b.lis
(the lidar and radiosonde data file list) and of course the lidar and 
radiosonde data themselves; para.dat has the following format:


\noindent
{\tt
npts,mpt,dpt,tpt,inpts,rrcor,lambda\\
100,3,3,5,20,1,1.064\\
rez,zero,nob,high,nlog,mixbit\\
40.00,-1.0e+20,1,1.0e+20,1,0\\
corrfak,xovlp,ieb,k,seth,asw,conh,avlen\\
0.0,800.0,401,1.0,0,1,1500.0,20\\
nd,gain,nos,multi,offset,fav\\
0.0,0.0,250,1.,1.0e-30,0\\
thrush,ntot,iabove,iwk,isat,baseline,add,plusig\\
100.,14,2,0,1,1,1,10.0\\
}

\noindent
Within the file each parameter required in the execution of the program
is listed below the abbreviation used as the variable in the program 
itself and represent the following:

\begin{itemize}
\item npts: number of points in the lidar data per profile. The current 
version allows the user to select either 100 or 200 points [integer].
\item mpt: number of points in the cloud base algorithm which are searched
for to locate the subsequent peak signal corresponding a given cloudbase
height in the cloud base height algorithm [integer].
\item dpt: number of points fit used in determining the derivative in the
cloud base height algorithm [integer].
\item tpt: number of range interval points averaged 
above and below the cloud in obtaining an optical 
depth and a mean optical extinction value
using the ratio method. The algorithm will average the number of
points available. Hence if the number of points above the cloud
in question is less than {\em tpt}, then  it will use the number of
points avaiable [integer].
\item inpts: number of points that the signal will be averaged above
the cloud in order to obtain an adequate signal-to-noise ratio
in the reference height data The algorithm will average the number of
points available. Hence if the number of points above the cloud
in question is less than {\em tpt}, then  it will use the number of
points avaiable [integer].
\item rrcor: selects between removing range correction from lidar data
(1) before processing or to not remove it (0) [integer].
\item lambda: this is the wavelength used in the lidar system given
in $\mu m$. This data is used to calculate the Rayleigh 
backscatter cross-section in the reference extinction coefficient
calculation \cite{mn}.
\item rez: vertical resolution of the lidar system [real].
\item zero: lowest possible value from the lidar data [real].
\item nob: number of individual profiles to average over in order to
obtain a sufficent SNR (no longer fully implemented, set to 1) [integer].
\item high: highest possible value the optical extinction can have. Used in
the inversion to account for cases where a zero occurs in
the denominator (must be $>>1$, ussually set to about $10^{+20}$) [real].
\item nlog:selects between using the logarithm of the backscatter signal
(1) or the linear value (0) in the cloud base algorithm. For I.R. data
set to 1 [integer].
\item mixbit: selects between correlating meteorological measurements
with cloud base(1), mid(0) or top (2) height [integer].
\item corrfak: factor which sets the rejection criteria in the cloud
base algorithm. This factor that determines the sensitivity of 
the algorithm. Normally low corrfak values tend to trigger on
noise spikes, whereas high corrfak will not recognize weak cloud layers. 
It is described in the previous chapter in greater detail. 
for our data we have obtained satisfactory results with corrfak set 
to 0. [real].
\item xovlp: overlap altitude height [height rejection criteria, for
region of initial transmitter-receiver field-of-view (FOV)
intersection.] (ussually set between 500-800) [real].
\item ieb: total number of complete profiles or traces to be read from the
lidar data file [integer].
\item k: backscatter to extinction ratio exponent, ussually set to 1.
\item seth: used in conjunction with the variable {\em conh}. When set to 1,
it instructs the program to begin inverting the data from {\em conh}
downwards [integer].
\item asw: When set to 1, this parameter instructs the program to 
average from the top most range interval down to and stopping at 
the height specified by {\em conh} described below. The averaging
is from point to point signals values adjacent in range 
over the number of intervals on either side of a given point
specified by {\em avlen} described below. If {\em fav} (described
below is set to 1, then a least squares fit is performed of
{em avlen} points (up to 5 points only advised) instead of averaging.
If set to (0) no averaging or fitting is performed [integer].
\item avlen: specifies the number of intervals on either side of a given
point that are averaged or fitted (see also {\em asw, avlen, fav}) [integer].
\item conh: the confidence height is the highest possible altitude at which the
user is confident of the signal to noise ratio. This setting 
specifies the height at which the program stops averaging
(starting at the topmost range height downwards) to terminate 
averaging signals adjacent in range or the range at which to begin
inverting when used with the parameters {\em asw, seth} or {\em fav}, 
as previously described [integer].
\item nd: neutral density (optical) filtering applied during collection
    of data [part of cloud base algorithm rejection criteria].
\item gain: currently not in use (set to 0).
\item nos: number of shots collected and hardware averaged per profile
(set to 250).
\item multi: can be set to any number that the user wishes to
mutiply the entire profile by. This is useful 
extremely small values of signal where computational
limits may interfere with optical inversions or cloud
basing height location algorithms. Since all factors are divided out
in the inversions, this value has no effect on our
optical derived qunatities [real].
\item offset:any constant that the user wishes to add to the entire 
profile [real].
\item fav: selects between least-sqaures fit (1) and simple averaging (0)
of the signal from the topmost range height down to the 
height specified by the parameter {\em conh}. 
Must be used with {\em asw} (see above) set to 1 [integer].
\item thrush: currently not in use (set to 0).
\item ntot: total number of points in the sonde data [integer].
\item iabove: number of points above the cloud top to set the 
reference height for the Klett Inversion as [integer].
\item iwk: selects between a standard implementation of the far end ({\em Klett}) 
(set to 0) a modified algorithm for $k=1$ described in the previous
chapter by the Eq.~{\ref{wolfee}} and suggested by W. Steinbrecht (set to 1)
and the {\em slope method} (set to 2) inversions for obtaining the
optical extinction profiles [integer].
\item isat: selects whether or not to invert (0) or to disregard and not
invert saturated data [integer].
\item baseline: selects whether or not to invert using an average of 
the actual signal determined by inpts (0) or to 
use the value specified by the parameter {\em zero}  as the value
for the reference signal (1). This is useful only if the signal value
at the reference height is known [integer].
\item add: selects between background addition of the lowest value (negative)
signal plus and offset specified above when on (1) or to not add
any constant to the signal (0) [integer].
\item plusig:user specified amount that can be added to the calculated Rayleigh
extinction value [real].
\end{itemize}

\section{KNMI2B2 Input/Output File Formats:}
  
KNMI2B2 outputs five data files the names of which are specified in the 
third to seventh lines of the file list known as knmi2b.lis. The first
two lines are input file names; knmi2b.lis has the following format:


\noindent
{\tt
93216riv.dat\\  
9321600.dat\\
93216clb.dat\\  
93216opt.dat\\
93216op2.dat\\
93216mix.dat\\
93216and.dat\\
93216tru.dat\\
93216nrc.dat\\
93216riv.t\\
}


\noindent
The first line specifies the lidar backscatter file to be read,
in the format: rows of 100 data points, 6 integer digit wide ASCII data seperated 
by single spaces between points.
\noindent
The second line indicates the name of the sonde file for that day,
in the format:

\begin{center}
 	{\tt 1x,pr,1x,ht,1x,tp,2x,rh,2x,dp,2x,wd,3x,wv}
\end{center}
\noindent
where,\\
\noindent
nx: n character space(s)\\
pr: pressure in millibars in real 6 digit wide, 1 decimal place format.\\
ht: height in metres in integer 5 digit wide format.\\
tp: temperature in degrees Celsius in real 5 digit wide, 1 decimal place format.\\
rh: relative humidity in percent in integer 2 digit wide format.\\
dp: dew-point temperature in real 5 digit wide, 1 decimal place format.\\
wd: wind direction in integer 3 digit wide format.\\
wv: wind velocity in real 4 digit wide, 1 decimal place format.\\

The third line specifies the name of the output file containing the
cloud base height algorithm \cite{srpwsaic} derived products in the format:

\begin{center}
cbh 1, prh 1, ath 1, cbh 2, prh 2, ath 2, cbh 3, prh 3, ath 3
\end{center}
\noindent
where,
\noindent
cbh x: cloud base height layer x\\
prh x: peak return height layer x\\
ath x: apparent top height layer x\\

All in metres and in real numbers format, 7 digit wide, no decimals.

The fourth line specifies the name of the output file containing all
the optical extinction profiles in rows of 100 data points 10 digits
wide, exponent with 3 decimal places.

The fifth line specifies the name of the output file containing the
derived optical properties in the format:

\begin{center}
	{\tt nl,op1,op2,op3,me1,me2,me3}
\end{center}
\noindent
nl: number of cloud layers detected in a given profile.\\
opn: optical depth for cloud layer n.\\
men: mean optical extinction for cloud layer n.\\

All derived from integrating the extinction profile 
generated using the Klett inversion algorithm over the cloud
layer \cite{jdk1}. 

The sixth line specifies the name of the output file containing the
meteorological parameters read in from the sonde data
correlated with the clouds optical and geometric properties at the cloud
base, mid or top height in the folowing format:

\begin{center}
		{\tt ht,1x,ct,1x,od,1x,me,2x,tp,1x,rh}
\end{center}
\noindent
nx: n character space(s)\\
ht: base, mid-cloud or top height in metres in real 7 digit wide, 
no decimal place format.\\
ct: cloud geometric thickness in metres in real 7 digit wide, 
    no decimal place format.\\
od: optical depth, in real 10 digit wide, 3 decimal place format.\\
me: mean optical extinction in 1/km, in real 10 digit wide, 
    3 decimal place format.\\
tp: temperature in degrees Celsius in real 8 digit wide, 
    1 decimal place format.\\
rh: relative humidity in percent in integer 4 digit wide format.\\

The seventh line specifies the name of the output file containing the
identical optical extinction profiles in rows of 100 data points 10 digits
wide, exponent with 3 decimal places specified in line four, but with 
each profile preceeded by cloud base height algorithm \cite{srpwsaic}
derived products in the
format described for the file specified third line for that given
profile.

The eigth line specifies the name of the output file containing the
derived optical properties in the format:

\begin{center}
	{\tt nl,op1,op2,op3,me1,me2,me3}
\end{center}
\noindent
nl: number of cloud layers detected in a given profile in a 2 wide integer.\\
opn: optical depth for cloud layer n in exponent 
     10 wide, 3 decimal place format.\\
men: mean optical extinction for cloud layer n in exponent 
     10 wide, 3 decimal place format.\\

All derived from ratioing the signal just above and below the
intervening cloud.

The ninth line specifies the name of the output file 
containing the non-range corrected version of the original backscatter
profile in the rows of 100 points formatted in real numbers, 
7 wide, no decimals.

The tenth and final line specifies the name of the time index file 
provided by RIVM. This file also contains the signal saturation
status flag. The program will skip the first 13 columns and read
the single character in column 14 which is either a j or a n.
j positive is and n means no saturation.

Although not necessary, it useful to adopt some sort of naming
convention to keep track of these different files as has been
demonstrated above.

\section{Installing KNMI2B2 and CirrusWare}

To install KNMI2B2 and the other CirrusWare software,
place the installation floppy disk in drive A
and type:

\begin{center}
	$>$A:$\backslash$setup.bat
\end{center}

The software will be automatically loaded into a newly created 
directory called CirrusWare. To create an applications window simply
run Windows, click on the file manager icon, locate TRY
and KNMIED2 in the CirrusWare directory and drag them to
the applications window or one that you have created for
CirrusWare. 

\section{Tutorial}

To become familiar with KNMI2B2 follow these steps:

If the package has been installed, and you are running 
the software from MS/DOS, change to the CirrusWare directory and ensure
that the following files are present:


\noindent
{\tt
9321600.DAT\\
93216RIV.DAT\\
93216RIV.T\\
KNMI2B2.FOR\\
KNMI2B2.EXE\\
KNMI2B2.LIS\\
PARA.DAT\\
TRY.BAT\\
KNMIED2.BAT\\
EMACS.EXE\\
EMACS.HLP\\
EMACS.RC\\
}

\noindent
The source code will be listed in KNMI2B2.FOR. To run the program, type

\begin{center}
	$>$try
\end{center}
\noindent
at the DOS prompt. If you are running it under Windows 3.0, find the
applications group:
\begin{center}
		CirrusWare
\end{center}
\noindent
and click on try.

This batch file will delete any existing output files if they exist
and open the file para.dat with a MicroEMACS editor. Change or adjust 
the parameters of interest as desired by moving the cursor, inserting 
text and using the delete keys as required. For more details on the
editor use the $<f6>$ key to expand the help window while it is running.

To use the editor for any other application simply type:
\begin{center}
		C:$\backslash$emacs $<filename>$
\end{center}
\noindent
Once the necssary changes have been made, type $<CTRL>$-X-S to save the
settings and $<CTRL>$-X-C to exit the editor and begin the program.

KNMI2B2 will display a short identifcation logo, followed by 
prompts pertaining to the trace it is currently analysing.
A TRACE ANALYZED will appear as the analysis is completed for each trace.
It will indicate that the process is complete when a 

\begin{center}
		...it is finally DONE 
\end{center}
\noindent
appears. Upon examining the directory you have been running in
you will find the following additional new files:


\noindent
{\tt 93216AND.DAT\\
93216CLB.DAT\\
93216MIX.DAT\\
93216NRC.DAT\\
93216OP2.DAT\\
93216OPT.DAT\\
93216TRU.DAT\\
93216PRO.DAT\\
}


These can now be graphically displayed using Origin as discussed
earlier. This can be repeated as many times as necessary in order
for you to become more familiar or optimize the program parameters
for a particular data set.

You may decide you want to modifiy some part of KNMI2B2 at some
point. In such case you can click on, in Windows, or type, in DOS:

\begin{center}		
			$>$KNMIED2
\end{center}

\noindent
This batch file launches the emacs editor to open the source
code for KNMI2B2. Onces edits have been made, saved and the file 
closed, and if the Microsoft MS-FORTRAN 5.1 Compiler is installed
on your system, it launches the Programmers Workbench. To create
the new executable, using the mouse, pull down the MAKE menu and 
click on MAKE KNMI2B2.

Once it is compiled and linked an executable will be created, if
the code passes without errors. Exit the compiler using the $<ALT>-<F4>$
combination. You can now run the modified program.

\section{Examples of KNMI2B2 Outputs:}

The following figures have been produced from data processed
by the the software package KNMI2B2.

The software  package Origin by MicroCal Software Inc. a Windows 3.0
application has been used to present the data here. 
Grey-scale plots require the use of the 3D version
of Origin. Please consult the documentation that accompanies
Origin and MicroCal for more details.

Figure~{3.4} is a grey-scale plot showing the backscatter
intensity time series of the lidar returns for the day 93216 (August 4th, 1993) 
without range correction (file 93216nrc.dat).

\begin{figure}
\vspace{5.0in}
\caption{Grey-scale plot showing the backscatter
intensity time series of the lidar returns for the day 93216 (August 4th, 1993) 
without range correction (files 93216nrc.dat)}
\end{figure}

Figure~{3.5} shows a plot of the file 93216clb.dat, indicating the
detected cloud base, peak signal return height and apparent cloud
top height for this day also.

\begin{figure}
\vspace{5.0in}
\caption{Plot of the file 93216clb.dat, indicating the
detected cloud base, peak signal return height and apparent cloud
top height for the day 93216 (August 4th, 1993)}
\end{figure}

Figure~{3.6}, is a grey-scale plot of the optical extinction derived
from the backscatter intensity for that day also, using the Klett
Inversion, it is generated from the file 93216opt.dat \cite{jdk1}.
Note the bands of high values along the top edge of the
time-series. These are the result of overestimates in the 
inversion boundary condition. As discussed it expected that 
the effect such errors will dissapate and the 
resultant profile should be accurate to within 10\% of 
the actual value below $1500 m$.

\begin{figure}
\vspace{5.0in}
\caption{Grey-scale plot of the optical extinction derived
from the backscatter intensity, for the day 93216 (August 4th, 1993), 
using the Klett Inversion}
\end{figure}

A time-series of the mean optical extinction and cloud optical depths
can be generated from the 93216op2.dat or 93216tru.dat files. The former
values are derived from the optical extinction profiles derived by using
the Klett Inversion \cite{jdk1} whereas 
the latter are values derived by comparing
the signal above and below the cloud {\em (ie. the ratio method)}
as discussed in the previous chapters.
By plotting the values of mean optical extinction or optical dpeth 
derived from both methods, against each other, the success of the
inversion can be gauged. See Figures~{3.7-3.8}.

\begin{figure}
\vspace{5.0in}
\caption{Time-series of the optical depth
generated from the 93216op2.dat file}
\end{figure}

\begin{figure}
\vspace{5.0in}
\caption{Scatter plot comparing the cloud 
optical depths generated from both 
93216op2.dat and 93216tru.dat files}
\end{figure}

Finally, plots of cloud optical or physical data correlated 
with meteorological data
measured by sonde that day can also be generated from the file
93216mix.dat. Figure~{3.9} shows
a scatter plot of cloud optical extinction is compared with temperature
at mid cloud height.

\begin{figure}
\vspace{5.0in}
\caption{Scatter plot of cloud optical extinction compared with temperature
at mid-cloud height}
\end{figure}

Comparable parameters in the file 93216mix.dat as stated previously
include: mid-cloud or base or top height, cloud geometric thickness,
opitcal depth, mean optical extinction, temperature and relative humidity.
